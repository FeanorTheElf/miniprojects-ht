\documentclass{scrartcl}

\usepackage{graphicx}
\usepackage[utf8]{inputenc}
\usepackage[T1]{fontenc}
\usepackage{lmodern}
\usepackage[english]{babel}
\usepackage{amsmath}
\usepackage{amsthm}
\usepackage{mathtools}
\usepackage{amssymb}
\usepackage{listings}
\usepackage{xparse}
\usepackage{geometry}
\usepackage{enumerate}
\usepackage{tikz}
\usepackage{thmtools}
\usepackage{hyperref}
\usepackage[style=english]{csquotes}
\usepackage[language=english, backend=biber, style=alphabetic, sorting=nyt]{biblatex}

\hypersetup{
    colorlinks,
    linkcolor={red!50!black},
    citecolor={blue!50!black},
    urlcolor={blue!80!black}
}

\usetikzlibrary{babel, positioning, shapes.geometric, arrows, arrows.meta}
\addbibresource{bibliography.bib}

\title{Miniproject - Elliptic Curves}
\author{Simon Pohmann}

\newcommand{\N}{\mathbb{N}}
\newcommand{\Z}{\mathbb{Z}}
\newcommand{\F}{\mathbb{F}}
\newcommand{\C}{\mathbb{C}}
\newcommand{\Q}{\mathbb{Q}}
\newcommand{\I}{\mathbb{I}}
\newcommand{\V}{\mathbb{V}}
\newcommand{\p}{\mathfrak{p}}
\newcommand{\Set}{\mathrm{\textbf{Set}}}
\newcommand{\Aff}{\mathrm{\textbf{Aff}}}
\newcommand{\Sch}{\mathrm{\textbf{Sch}}}
\newcommand{\Ring}{\mathrm{\textbf{Ring}}}
\newcommand{\Ab}{\mathrm{\textbf{Ab}}}
\newcommand{\Top}{\mathrm{Top}}
\newcommand{\Spec}{\mathrm{Spec}}
\newcommand{\Quot}{\mathrm{Quot}}
\newcommand{\im}{\mathrm{im}}
\renewcommand{\O}{\mathcal{O}}
\DeclareMathOperator*{\colim}{colim}
\newcommand{\notdivides}{\ \nmid \ }
\newcommand{\divides}{\ \mid \ }

\newcommand\restr[2]{{
    \left.\kern-\nulldelimiterspace
    #1
    \vphantom{\big|}
    \right|_{#2}
}}

\theoremstyle{definition}

\newtheorem{prop}[subsection]{Proposition}
\newtheorem{theorem}[subsection]{Theorem}
\newtheorem{lemma}[subsection]{Lemma}
\newtheorem{corollary}[subsection]{Corollary}

\newtheorem{problem}[subsection]{Problem}
\newtheorem{alg}[subsection]{Algorithm}
\newtheorem{definition}[subsection]{Definition}
\newtheorem{example}[subsection]{Example}
\newtheorem{remark}[subsection]{Remark}

\begin{document}
\maketitle
\listoftheorems

\section{Question 1}

\begin{example}[1(i)]
    Have
    \begin{equation*}
        |463^2 + 6|_5 = |214375|_5 = |5^4 \cdot 7^3|_5 = 5^{-4} < 5^{-3}
    \end{equation*}
    To find it, note that $|2^2 + 6|_5 < 1$ and use Newton's method.
    Set $x_0 = 2$ and have
    \begin{align*}
        x_1 = x_0 - \frac {x_0^2 + 6} {2x_0} = 2 - \frac {10} {4} = -\frac 1 2  \\
        x_2 = -\frac 1 2 - \frac {1/4 + 6} {-1} = \frac {25} 4 - \frac 1 2 = \frac {23} 4
    \end{align*}
    and indeed, $|(23/4)^2 + 6|_5 = |625/16|_5 = 5^{-4}$.
    Since the valuation $|\cdot|_5$ is non-Archimedian, observe that $|x^2 + 6|_5 < 5^{-3}$ holds for all $x \in \Q$ with $|x - 32/4|_5 = |4x - 32|_5 < 5^{-3}$.
    Hence, we look for $x \in \Z$ such that $5^4 \divides 4x + 23$.
    In other words, find $k \in \Z$ with $4 \divides k 5^4 - 23$, i.e. $k - 3 \equiv 0 \mod 4$.
    We find $k = 3$ and so $x = 463$.
\end{example}

\begin{example}[1(ii)]
    Let
    \begin{equation*}
        \alpha = 5^{-1} + 2 \cdot 5^0 + 5 (1 + 4 \cdot 5) \sum_{n \geq 0} 5^{2n} \in \Q_5
    \end{equation*}
    Note that in $\Q_5$ we have
    \begin{equation*}
        \sum_{n \geq 0} 5^{2n} = \sum_{n \geq 0} 25^n = \frac 1 {1 - 25} = -\frac 1 {24}
    \end{equation*}
    So
    \begin{equation*}
        \alpha = \frac 1 5 + 2 + 5 (21) \frac 1 {24} = \frac {263} {40}
    \end{equation*}
\end{example}
For the next exercises, we will slightly abuse notation and write
\begin{equation*}
    E(R) := \{ (x, y) \in E(\bar{k}) \ | \ x, y \in R \} \cup \{\O\}
\end{equation*}
for an Elliptic Curve $E$ defined over $k$ and any ring $R$ contained in the algebraic closure $\bar{k}$.
Note that this is usually not a group anymore, and does not have a lot of nice structure.
\begin{prop}[1(iii)]
    Consider the Elliptic Curve $E: y^2 = x^3 + 2x + 2$ defined over $\Z$.
    Then $E(\Z) = \{ \O \}$ and
    \begin{equation*}
        E(\Z_p) \neq \{ \O \} \ \Leftrightarrow \ p \neq 3
    \end{equation*}
\end{prop}
\begin{proof}
    First show that there exists some $(x, y) \in \tilde{E}(\F_p)$ with $y \neq 0$ for all primes $p \neq 3$.

    If $p \equiv 1, 5 \mod 8$, then $-1$ is a square in $\F_p$, thus there is $\alpha \in \F_p$ with $\alpha^2 = -1$ and so $(-1, \alpha) \in \tilde{E}(\F_p)$.
    If $p \equiv 7 \mod 8$, then (the law of Quadratic Reciprocity, e.g. \cite[Prop. I.8.6]{neukirch}) it follows that $2$ is a square in $\F_p$.
    Thus there is $\alpha \in \F_p$ with $\alpha^2 = 2$ and so $(0, \alpha) \in \tilde{E}(\F_p)$.

    Hence, consider now the case $p \equiv 3 \mod 8$.
    Note that
    \begin{equation*}
        \Delta(E) = 4 \cdot 2^3 + 27 \cdot 2^2 = 140 = 2^2 \cdot 5 \cdot 7
    \end{equation*}
    Hence we see that $p \notdivides \Delta(E)$ and so $\tilde{E}$ is an Elliptic Curve defined over $\F_p$.
    Now the Hasse bound \cite[Thm 1.15]{lecture} shows that
    \begin{equation*}
        \#\tilde{E}(\F_p) \in [p + 1 - 2\sqrt{p}, p + 1 + 2\sqrt{p}]
    \end{equation*}
    Note that for $p > 9$ have $\sqrt{p} < p/3$ and thus
    \begin{equation*}
        p + 1 - 2\sqrt{p} > 4
    \end{equation*}
    Thus $\tilde{E}(\F_p) \geq 5$ and so there must be $(x, y) \in \tilde{E}(\F_p)$ with $y \neq 0$, as there are at most four points on $\tilde{E}(\F_p)$ that do not satisfy this ($\O$ and possibly $(\alpha_i, 0)$ with $\alpha_i$ a root of $x^3 + 2x + 2$).

    Now consider any prime $p \neq 2, 3$ and $(\tilde{x}, \tilde{y}) \in \tilde{E}(\F_p), x, y \in \Z, \tilde{y} \neq 0$.
    Let $f(t) := t^2 - x^3 - 2x - 2$.
    Then
    \begin{equation*}
        |f(y)|_p \leq p^{-1} \quad \text{and} \quad |f'(y)|_p = |y|_p = 1
    \end{equation*}
    Thus $|f(y)|_p < |f'(y)|_p^2$ and Hensel's Lemma yields a root $\gamma \in \Z_p$ with $(x, \gamma) \in E(\Z_p)$.

    In the case $p = 2$, note that $f(t) := t^2 - 5^3 - 2 \cdot 5 - 2 = t^2 - 137$ satisfies
    \begin{equation*}
        |f(1)|_2 = |-136|_2 = |-17 \cdot 2^3|_2 = 2^{-3} < \left( 2^{-1} \right)^2 = |2|_2^2 = |f'(1)|_2^2
    \end{equation*}
    and so Hensel's Lemma \cite[Thm 2.14]{lecture} yields a point $(5, \gamma) \in E(\Z_2)$.

    The only remaining case is $p = 3$, and a trying all 9 points in $\F_3^2$ shows that $\tilde{E}(\F_3) = \{\O\}$.
    This clearly shows that $E(\Z_3) = \{ \O \}$ and so $E(\Z) = \{\O\}$.
\end{proof}
For the next exercise, we first summarize the techniques we have used above.
\begin{prop}[Existence of points over $\Z_p$]
    \label{prop:p_adic_points_techniques}
    Let $E: y^2 = x^3 + a_2 x^2 + a_4 x + a_6$ be an Elliptic Curve defined over $\Z$.
    Let $p$ be a prime.
    Then
    \begin{itemize}
        \item If $E(\Z_p) \neq \{\O\}$ then $\tilde{E}(\F_p) \neq \{\O\}$.
        \item Suppose $p \neq 2$. There is $(x, y) \in \tilde{E}(\F_p)$ with $y \neq 0$ if and only if there is $(x, y) \in E(\Z_p)$ with $|y|_p = 1$.
        \item Suppose $p \neq 2$. If $\#\tilde{E}(\F_p) \geq 5$ then there is $(x, y) \in E(\Z_p)$ with $|y|_p = 1$.
        \item Suppose $p \geq 11$ and $p \notdivides \Delta(E)$. Then there is $(x, y) \in E(\Z_p)$ with $|y|_p = 1$.
    \end{itemize}
\end{prop}
\begin{proof}
    The first part is trivial and follows from the fact that any $(x, y) \in E(\Z_p)$ yields $(\tilde{x}, \tilde{y}) \in \tilde{E}(\F_p)$.

    For the second part, note that by assumption, there is $(\tilde{x}, \tilde{y}) \in \tilde{E}(\F_p), x, y \in \Z$ with $|y|_p = 1$ and so
    \begin{equation*}
        |y^2 - x^3 - a_2 x^2 - a_4 x - a_6|_p \leq p^{-1} < 1 = 1^2 = |2y|_p
    \end{equation*}
    Hensel's Lemma now \cite[Thm 2.14]{lecture} shows that there is $\gamma \in \Z_p$ such that $\gamma^2 = x^3 + a_2 x^2 + a_4 x + a_6$ and so $(x, \gamma) \in E(\Z_p)$.
    Since $|y|_p = 1$ clearly also $|\gamma|_p = 1$.
    The other direction is obvious and follows directly by taking the reduction modulo $p$.

    For the third part, notice that there are at most three different points $(x, y) \in \tilde{E}(\F_p)$ with $y = 0$, as in this case $x$ is a root of the cubic $t^3 + a_2 t^2 + a_4 t + a_6$.
    Thus, if $\#\tilde{E}(\F_p) \geq 5$, there must be $(x, y) \in \tilde{E}(\F_p)$ with $y \neq 0$ and so the claim follows by the second part.

    For the fourth part, note that as above, $p > 9$ implies $\sqrt{p} < p/3$ and so the Hasse bound \cite[Thm 1.15]{lecture} yields (since $\tilde{E}$ is an Elliptic Curve by assumption, as $p \notdivides \Delta(E)$)
    \begin{equation*}
        \#\tilde{E}(\F_p) \geq p + 1 - 2\sqrt{p} > 4
    \end{equation*}
    thus $\#\tilde{E}(\F_p) \geq 5$. 
    The claim now follows by the third part.
\end{proof}
This already shows that we do not have to worry to much about the condition $E(\Z_p) \neq \{\O\}$ for $p \neq 2, 3, 5, 7$ prime, as we expect that it is fulfilled quite often.
My next try was to characterize in which cases there is no $(x, y) \in E(\Z_p)$, $|y|_p = 1$.
However it turns out that this never happens simultaneously for $p \in \{ 3, 5, 7 \}$ (which was how I understood the question at first).
On the other hand, I also found the following strengthening of the previous statement that completely finishes the case $p \geq 11$.
\begin{prop}
    \label{prop:points_in_EZp}
    Let $p \geq 11$ be a prime and $E: y^2 = x^3 + f_2 x^2 + f_1 x + f_0$ an Elliptic Curve with $f_0, f_1, f_2 \in \Z$.
    Then there is $(x, y) \in E(\Z_p)$ with $|y|_p = 1$.
\end{prop}
\begin{proof}
    If $p \geq 11$ and $p \notdivides \Delta(E)$ then $\tilde{E}$ is an Elliptic Curve over $\F_p$ and the claim follows from Proposition~\ref{prop:p_adic_points_techniques}.
    
    So assume now that $p \divides \Delta(E)$, hence $x^3 + f_2 x^2 + f_1 x + f_0$ factors as
    \begin{equation*}
        x^3 + \tilde{f}_2 x^2 + \tilde{f}_1 x + \tilde{f}_0 \equiv (x - \alpha)^2(x - \beta)
    \end{equation*}
    with $\alpha, \beta \in \bar{\F}_p$.
    However, note that $\F_p$ is perfect, so $(x - \alpha)^2(x - \beta)$ cannot be irreducible over $\F_p$, otherwise $\F_p[x] / \langle(x - \alpha)^2(x - \beta)\rangle$ would be a non-separable field extension of $\F_p$.
    Thus, either $\alpha \in \F_p$ or $\beta \in \F_p$.
    If $\alpha \in \F_p$, then clearly also $\beta = -2\alpha - \tilde{f}_2 \in \F_p$.
    If $\beta \in \F_p$, then also $(x - \alpha)^2 \in \F_p[x]$ and again by perfectness of $\F_p$, note that $\alpha \in \F_p$.
    So $\alpha, \beta \in \F_p$.

    Now note that for $t \in \F_p$ have
    \begin{equation*}
        \left( t^2 + \beta, \ t (t^2 + \beta - \alpha) \right) \in \tilde{E}
    \end{equation*}
    Hence, we find a function
    \begin{equation*}
        \phi: \F_p \to \tilde{E}(\F_p) \setminus \{\O\}, \quad t \mapsto \left( t^2 + \beta, \ t (t^2 + \beta - \alpha) \right)
    \end{equation*}
    If there is $\gamma \in \F_p$ with $\gamma^2 = \alpha - \beta$, then 
    \begin{equation*}
        \restr{\phi}{\F_p \setminus \{-\gamma\}}: \F_p \setminus \{-\gamma\} \to \tilde{E}(\F_p)
    \end{equation*}
    is injective, otherwise $\phi$ is injective.
    Hence, we see that $\#(\tilde{E}(\F_p) \setminus \{\O\}) \geq \#\F_p - 1 \geq 4$ and so $\#\tilde{E}(\F_p) \geq 5$.
    It follows that there is $(\tilde{x}, \tilde{y}) \in \tilde{E}(\F_p)$ with $\tilde{y} \neq 0$.
    By a Hensel-lifting argument as in Proposition~\ref{prop:p_adic_points_techniques}, we now see that there is $\gamma \in \Z_p$ with $(x, \gamma) \in E(\Z_p)$ and $|\gamma|_p = 1$.
\end{proof}
The above proposition shows that constructing Elliptic Curves $E: y^2 = x^3 + f_2 x^2 + f_1 x + f_0$ such that there is $(x, y) \in E(\Z_p)$ with $|y|_p = 1$ for all primes $p \neq 3, 5, 7$ is indeed quite simple, as almost all curves satisfy this.
This only case that can fail is $p = 2$, but here, the condition is fulfilled quite often, so we can just try different choices.
Using a small brute force search then yields the following examples.
\begin{example}[1(iv)]
    Let
    \begin{align*}
        E_1: \ &y^2 = x^3 + 2 x \\
        E_2: \ &y^2 = x^3 + 2 x^2 + 6 x + 5\\
        E_3: \ &y^2 = x^3 + 6x + 1
    \end{align*}
    Note that
    \begin{align*}
        1^2 &\equiv 3^3 + 2 \cdot 3 = 33 \mod 8 \\
        1^2 &\equiv 2^3 + 2 \cdot 2^2 + 6 \cdot 2 + 5 = 33 \mod 8 \\
        1^2 &\equiv 4^3 + 6 \cdot 4 + 1 = 89 \mod 8
    \end{align*}
    so Hensel's Lemma yields \cite[Thm 1.15]{lecture} points $(x, y) \in E_i(\Z_2)$ with $|y|_2 = 1$ for $i \in \{1, 2, 3\}$.
    By Proposition~\ref{prop:points_in_EZp}, we have points $(x, y) \in E(\Z_p)$ with $|y|_p = 1$ for all $p \geq 11$.

    Finally, note that trying all points shows
    \begin{align*}
        \tilde{E}_1(\F_3) &= \{ (0, 0), (1, 0), (2, 0), \O \} \\
        \tilde{E}_1(\F_5) &= \{ (0, 0), \O \} \\
        \tilde{E}_2(\F_7) &= \{ (1, 0), (5, 0), (6, 0), \O \}
    \end{align*}
    so there is no point $(x, y) \in E_i(\Z_p)$ with $|y|_p = 1$ for $p \in \{3, 5, 7\}$ and a suitable $i$. 
\end{example}

\section{Question 2}
\begin{example}[2(i)]
    Consider the Elliptic Curve $E: y^2 = x(x + 1)(x + 4)$ defined over $\Q$.
    Note that the reduction $\tilde{E}$ modulo 5 is still an Elliptic Curve, as the roots $0, 1, 4$ are distinct modulo 5.
    By \cite[Lemma 5.1]{lecture}, there is an embedding
    \begin{equation*}
        E_{\mathrm{tors}}(\Q) \hookrightarrow \tilde{E}(\F_5)
    \end{equation*}
    Note that
    \begin{equation*}
        \tilde{E}(\F_5) = \{ (0, 0), (1, 0), (2, 1), (2, 4), (3, 2), (3, 3), (4, 0), \O \}
    \end{equation*}
    has order 8.
    Clearly
    \begin{equation*}
        (0, 0), (-1, 0), (-4, 0), \O \in E_{\mathrm{tors}}(\Q)
    \end{equation*}
    So the only remaining question is whether this is all the torsion (i.e. $\#E_{\mathrm{tors}}(\Q) = 4$) or there are more points (i.e. $\#E_{\mathrm{tors}}(\Q) = 8$).

    Consider now $P = (-2, 2) \in E$.
    The tangent at $P$ is given by $y = -x$ and the third point of intersection with $E$ is thus $(0, 0)$.
    Hence $P + P = (0, 0)$ and so $[4]P = \O$.
    It follows that $\#E_{\mathrm{tors}}(\Q) = 8$ and furthermore that 
    \begin{equation*}
        E_{\mathrm{tors}}(E) = \langle P, (-1, 0) \rangle \cong \Z/4\Z \oplus \Z/2\Z
    \end{equation*}
\end{example}
\begin{example}[2(ii)]
    \label{ex:2ii}
    Consider the Elliptic Curve $E: y^2 = x(x + 1)(x - 8)$.
    Note that we have an isomorphism
    \begin{equation*}
        \psi: E \to E', \quad (x, y) \mapsto \left( x - \frac 7 3, y \right)
    \end{equation*}
    to the Elliptic Curve $E': y^2 = x^3 - \frac {73} 3 x - \frac {1190} {27}$ defined over $\Q$.
    Have that
    \begin{equation*}
        \Delta(E) = -72^2 = -5184 = \Delta(E')
    \end{equation*}
    Note that this has only the prime factors $2$ and $3$.
    As before, this shows that all the results from the lecture on the reduction modulo $p \neq 2, 3$ are also valid for the curve $E$, even though it is not defined by an equation of the form $y^2 = x^3 + Ax + B$.
    We see that
    \begin{equation*}
        \tilde{E}(\F_7) = \{ (0, 0), (1, 0), (4, 2), (4, 5), (5, 1), (5, 6), (6, 0), \O \}
    \end{equation*}
    and thus has order 8.
    As before, we this only leaves two possible cases, either the obvious 2-torsion points are all torsion points (i.e. $\#E_{\mathrm{tors}}(\Q) = 4$) or each of the points $\tilde{E}(\F_5)$ lifts to a torsion point (i.e. $\#E_{\mathrm{tors}}(\Q) = 8$).

    Unlike the previous example however, this time the former is the case.
    To see this, we use the Nagell-Lutz theorem \cite[5.4]{lecture}.
    Assume $(x, y) \in E_{\mathrm{tors}}(\Q)$ with $y \neq 0$.
    Then it yields that $y^2 \divides \Delta(E) = -72^2$ and so $y \divides 72$.
    So
    \begin{equation*}
        y \in \{ \pm 1, \pm 2, \pm 4, \pm 8, \pm 3 \pm 6, \pm 12, \pm 24, \pm 9, \pm 18, \pm 36, \pm 72 \}
    \end{equation*}
    Furthermore $y \not\equiv 0 \mod 7$ and since $(\tilde{x}, \tilde{y}) \in \tilde{E}(\F_7)$, it follows that
    \begin{equation*}
        (\tilde{x}, \tilde{y}) \in \{ (4, \pm 2), (5, \pm 1) \} \ \Rightarrow \ \tilde{y} \in \{ \pm 1, \pm 2 \}
    \end{equation*}
    Thus we only have the possibilities
    \begin{equation*}
        y \in \{ \pm 1, \pm 2, \pm 8, \pm 6, \pm 12, \pm 9, \pm 36, \pm 72 \}
    \end{equation*}
    Furthermore, observe that
    \begin{equation*}
        \tilde{E}(\F_{11}) = \{ (0, 0), (5, 3), (5, 8), (6, 2), (6, 9), (8, 0), (10, 0), \O \}
    \end{equation*}
    and so it follows by the same argument that
    \begin{equation*}
        \tilde{y} \in \{ \pm 2, \pm 3 \}
    \end{equation*}
    This further restricts the possibilities to
    \begin{equation*}
        y \in \{ \pm 2, \pm 8, \pm 12, \pm 9 \}
    \end{equation*}
    Finally, observe that none of the equations
    \begin{align*}
        4 &= x^3 - 7x^2 - 8x \\
        64 &= x^3 - 7x^2 - 8x \\
        144 &= x^3 - 7x^2 - 8x \\
        81 &= x^3 - 7x^2 - 8x
    \end{align*}
    has a solution in $\Q$.
    To see this, use e.g. the rational root theorem and some computation:

    The only factors of $4$ are $\pm 1, \pm 2, \pm 4$ and none solves $4 = x^3 + 7x^2 - 8x$.
    The only factors of $64$ are $\pm 1, \pm 2, \pm 4, \pm 8, \pm 16, \pm 32, \pm 64$ and none solves $64 = x^3 - 7x^2 - 8x$.
    The only factors of $144$ are $\pm 1, \pm 2, \pm 4, \pm 8, \pm 16, \pm 3, \pm 6, \pm 12, \pm 24, \pm 48, \pm 9, \pm 18, \pm 36, \pm 72, \pm 144$ and none solves $144 = x^3 - 7x^2 - 8x$.
    The only factors of $81$ are $\pm 1, \pm 3, \pm 9, \pm 27, \pm 81$ and none solves $81 = x^3 - 7x^2 - 8x$.

    Note that the usual approach to bound the size of $E_{\mathrm{tors}}(\Q)$ is to use the theorem that this embeds into $\tilde{E}(\F_p)$ whenever $\tilde{E}$ is an Elliptic Curve.
    However, for this example, this was not sufficient, as we could not find a prime such that the group $\Z/4\Z \oplus \Z/2\Z$ does not embed into $\tilde{E}(\F_p)$.
    In the next part, we want to study this phenomenon more carefully and indeed see that there is no such prime, i.e. it is impossible to show that $\#E_{\mathrm{tors}}(\Q) \neq 8$ by just considering the reductions modulo $p$.
\end{example}
First, it is convenient to have a closed formula for the $x$-coordinate of $[2]P$ for a point $P$ on an Elliptic Curve.
\begin{prop}[Duplication Formula]
    \label{prop:duplication_formula}
    Let $E: y^2 = x^3 + a_2 x^2 + a_4 x + a_6$ be an Elliptic Curve over a field $k$.
    For a point $P \in E$ with $P \neq \O$ denote by $x(P)$ its (affine) $x$-coordinate.
    Then have for all $P \in E$ with $P \neq -P$ that
    \begin{equation*}
        x([2]P) = \frac {x(P)^4 - 2a_4 x(P)^2 - 8a_6x(P) + a_4^2 - 4a_2a_6} {4(x(P)^3 + a_2 x(P)^2 + a_4 x(P) + a_6)}
    \end{equation*}  
\end{prop}
\begin{proof}
    Consider the tangent at $P = (a, b)$.
    Differentiating the equation of $E$ gives
    \begin{equation*}
        2y\frac {dy} {dx} = 3x^2 + 2a_2x + a_4
    \end{equation*}
    so it has slope
    \begin{equation*}
        \lambda = \frac {3a^2 + 2a_2a + a_4} {2b}
    \end{equation*}
    and the equation $y = \lambda (x - a) + b$.
    Note that after plugging this into the equation for $E$, the quadratic term has the coefficient $a_2 - \lambda^2$, so
    \begin{align*}
        x([2]P) &= \lambda^2 - a_2 - 2 x(P) = \frac {(3x(P)^2 + 2a_2x(P) + a_4)^2} {4b^2} - a_2 - 2 x(P) \\
        &= \frac {(3x(P)^2 + 2a_2x(P) + a_4)^2} {4(x(P)^3 + a_2 x(P)^2 + a_4 x(P) + a_6)} - a_2 - 2 x(P)
    \end{align*}
    Expanding this yields the claimed expression.
\end{proof}
\begin{prop}[Reductions mod $p$ are not enough]
    \label{prop:subgroup_mod_every_p}
    Let $E: y^2 = x(x + 1)(x - 8)$ be the Elliptic Curve from the previous example.
    The for each prime $p \geq 5$, have that $\Z/4\Z \oplus \Z/2\Z$ is a subgroup of $\tilde{E}(\F_p)$.
\end{prop}
\begin{proof}
    First of all, note that the duplication formula from Proposition~\ref{prop:duplication_formula} has the form
    \begin{equation*}
        x([2]P) = \frac {x(P)^4 + 16 x(P) + 64} {4 x(P)^3 - 28 x(P)^2 - 32 x(P)}
    \end{equation*}
    Consider any prime $p \geq 5$.

    \paragraph{Case 1} If $-1$ is a quadratic residue modulo $p$, then there is $\beta \in \F_p$ with $\beta^2 = -36$.
    Have then that $(2, \beta) \in \tilde{E}(\F_p)$ and
    \begin{equation*}
        x([2](2, \beta)) = \frac {2^4 + 16 \cdot 2^2 + 64} {4 \cdot 2^3 - 28 \cdot 2^2 - 32 \cdot 2} = \frac {144} {-144} = -1
    \end{equation*}
    and so $[2](2, \beta) = (-1, 0)$ is a 2-torsion point.
    Thus $(2, \beta)$ has order $4$ and we see that
    \begin{equation*}
        \langle (2, \beta), (0, 0) \rangle \cong \Z/4\Z \oplus \Z/2\Z
    \end{equation*}

    \paragraph{Case 2} If $-2$ is a quadratic residue modulo $p$, then there is $\alpha \in \F_p$ with $\alpha^2 = -8$.
    Then
    \begin{equation*}
        (\alpha - 8)^2 = (\alpha^2 + \alpha)(\alpha - 8) = \alpha(\alpha + 1)(\alpha - 8)
    \end{equation*}
    With $\beta := \alpha - 8$ we now find $(\alpha, \beta) \in \tilde{E}(\F_p)$ and
    \begin{equation*}
        x([2](\alpha, \beta)) = \frac {\alpha^4 + 16 \alpha^2 + 64} {4 \alpha^3 - 28 \alpha^2 - 32 \alpha} = \frac {(\alpha^2 + 8)^2} {4 \alpha^3 - 28 \alpha^2 - 32 \alpha} = 0
    \end{equation*}
    and so $[2](\alpha, \beta) = (0, 0)$ is a 2-torsion point.
    Hence, $(\alpha, \beta)$ has order $4$ and thus
    \begin{equation*}
        \langle (\alpha, \beta), (-1, 0) \rangle \cong \Z/4\Z \oplus \Z/2\Z
    \end{equation*}

    \paragraph{Case 3} If $2$ is a quadratic residue modulo $p$, then there is $\alpha' \in \F_p$ with $(\alpha')^2 = 72$ and so there is $\alpha = \alpha' + 8$ with $\alpha^2 - 16\alpha - 8 = 0$.
    Note that $\alpha^2 = 16\alpha + 8$ and thus
    \begin{equation*}
        (9\alpha - 24)^2 = 81 \cdot 16 \alpha + 81 \cdot 8 - 432 \alpha + 576 = 1224 + 864\alpha = \alpha^3 - 7\alpha^2 - 8\alpha= \alpha (\alpha + 1) (\alpha - 8)
    \end{equation*}
    With $\beta := 9\alpha - 24$ we now find $(\alpha, \beta) \in \tilde{E}(\F_p)$ and
    \begin{align*}
        x([2](\alpha, \beta)) =& \frac {\alpha^4 + 16 \alpha^2 + 64} {4\alpha^3 - 28\alpha^2 - 32\alpha} = \frac {(\alpha^2 + 8)^2} {4\alpha(\alpha + 1)(\alpha - 8)} \\
        =& \frac {16^2(\alpha + 1)^2} {4\alpha(\alpha + 1)(\alpha - 8)} = \frac {64 (\alpha + 1)} {(\alpha^2 - 8\alpha)} = \frac {64 (\alpha + 1)} {16\alpha + 8 - 8\alpha} = 8
    \end{align*}
    and so $[2](\alpha, \beta) = (8, 0)$ is a 2-torsion point.
    Hence $(\alpha, \beta)$ has order $4$ and thus
    \begin{equation*}
        \langle (\alpha, \beta), (0, 0) \rangle \cong \Z/4\Z \oplus \Z/2\Z
    \end{equation*}

    Since the Legendre symbol is multiplicative and $(-2)(-1) = 2$, these cases are exhaustive.
\end{proof}
To find more examples, it might be a good idea to use the structure from the previous theorem, but take another set of exhaustive cases.
So consider an Elliptic Curve
\begin{equation*}
    E: y^2 = x(x - \alpha)(x - \beta) = x^3 - (\alpha + \beta)x^2 + \alpha \beta x
\end{equation*}
with 3 nontrivial torsion points $(\alpha, 0), (\beta, 0), (0, 0)$ over $\Q$.
We study in which cases there is some $P \in \tilde{E}(\F_p)$ of order 4.
\begin{lemma}
    \label{prop:condition_z4_subgroup_mod_p}
    Let $E: y^2 = x(x - \alpha)(x - \beta)$ be an Elliptic Curve over a field $k$ of characteristic $\neq 2$.
    Then there exists $P \in E(k)$ of order 4 if and only if at least one of the following is true
    \begin{itemize}
        \item there is $\gamma \in k$ with $\gamma^2 = \alpha\beta$ and $2\gamma - \alpha - \beta$ is square in $k$
        \item there is $\gamma \in k$ with $\gamma^2 = \alpha(\alpha - \beta)$ and $2\gamma + 2\alpha - \beta$ is square in $k$
        \item there is $\gamma \in k$ with $\gamma^2 = \beta(\beta - \alpha)$ and $2\gamma + 2\beta - \alpha$ is square in $k$
    \end{itemize}
\end{lemma}
\begin{proof}
    The duplication formula for $E$ gives with $x = x(P)$ that
    \begin{equation*}
        x([2]P) = d(x) := \frac {x^4 - 2 \alpha \beta x^2 + \alpha^2 \beta^2} {4x^3 - 4(\alpha + \beta) x^2 + 4 \alpha \beta x}
    \end{equation*}

    \paragraph{Case 1} By assumption, there is $\gamma, \mu \in k$ with $\gamma^2 = \alpha\beta$ and $\mu^2 = 2\gamma + \alpha + \beta$.
    Thus
    \begin{equation*}
        d(\gamma) = \frac {\gamma^4 - 2\alpha\beta\gamma^2 + \alpha^2\beta^2} {4\gamma^3 - 4(\alpha + \beta) \gamma^2 + 4 \alpha \beta \gamma} = \frac 0 {4\gamma^3 - 4(\alpha + \beta) \gamma^2 + 4 \alpha \beta \gamma} = 0
    \end{equation*}
    Note further that
    \begin{equation*}
        \gamma^3 - (\alpha + \beta)\gamma^2 + \alpha\beta\gamma = 2\alpha\beta\gamma - \alpha\beta(\alpha + \beta) = \alpha\beta(2\gamma - \alpha - \beta) = \gamma^2 \mu^2
    \end{equation*}
    So there is a point $(\gamma, \gamma\mu) \in E(k)$ with $[2](\gamma, \mu) = (0, 0)$.

    \paragraph{Case 2} By assumption, there is $\gamma_0, \mu \in k$ with $\gamma_0^2 = \alpha(\alpha - \beta)$ and $\mu^2 = 2\gamma_0 + 2\alpha - \beta$.
    Let $\gamma := \alpha + \gamma_0$.
    Then note that $\gamma^2 - 2\alpha\gamma + \alpha\beta = 0$.
    Thus
    \begin{align*}
        &\gamma^4 - 2 \alpha \beta \gamma^2 + \alpha^2 \beta^2 = 4\alpha \gamma^3 - 4\alpha(\alpha + \beta) \gamma^2 + 4 \alpha^2 \beta \gamma \\
        =& \gamma^4 - 4\alpha \gamma^3 + (4\alpha^2 + 4\alpha\beta - 2\alpha\beta)\gamma^2 - 4\alpha^2\beta \gamma + \alpha^2\beta^2 \\
        =& \gamma^4 - 4\alpha \gamma^3 + 2\alpha(2\alpha + \beta)\gamma^2 - 4\alpha^2\beta \gamma + \alpha^2\beta^2 \\
        =& (\gamma^2 - 2\alpha \gamma + \alpha \beta)^2 = 0^2 = 0
    \end{align*}
    and so
    \begin{equation*}
        \gamma^4 - 2 \alpha \beta \gamma^2 + \alpha^2 \beta^2 = \alpha(4\gamma^3 - 4(\alpha + \beta) \gamma^2 + 4 \alpha \beta \gamma)
    \end{equation*}
    It follows that
    \begin{equation*}
        d(\gamma) = \frac {\gamma^4 - 2 \alpha \beta \gamma^2 + \alpha^2 \beta^2} {4\gamma^3 - 4(\alpha + \beta) \gamma^2 + 4 \alpha \beta \gamma} = \alpha
    \end{equation*}
    Furthermore note that
    \begin{align*}
        \gamma^3 - (\alpha + \beta)\gamma^2 + \alpha\beta\gamma =& \gamma(2\alpha\gamma - \alpha\beta) - (2\alpha\gamma - \alpha\beta)(\alpha + \beta) + \alpha\beta\gamma \\
        =& 2\alpha(2\alpha\gamma - \alpha\beta) - 2\alpha^2\gamma - 2\alpha\beta\gamma + \alpha^2\beta + \alpha\beta^2 \\
        =& \gamma(4\alpha^2 - 2\alpha^2 - 2\alpha\beta) + \alpha^2\beta + \alpha\beta^2 - 2\alpha^2\beta \\
        =& 2\alpha\gamma(\alpha - \beta) + \alpha\beta(\beta - \alpha) \\
        =& \alpha(\alpha - \beta)(2\gamma - \beta) \\
        =& (\gamma - \alpha)^2\mu^2
    \end{align*}
    So there is a point $(\gamma, (\gamma - \alpha)\mu) \in E(k)$ with $[2](\gamma, (\gamma - \alpha)\mu) = (\alpha, 0)$.

    \paragraph{Case 3} Exactly as in the previous case, by swapping $\alpha$ and $\beta$.

    The direction $\Leftarrow$ follows by distinguishing the cases $[2]P = (0, 0)$, $[2]P = (\alpha, 0)$ and $[2]P = (\beta, 0)$ and then reversing the above computation.
\end{proof}
\begin{lemma}
    \label{prop:condition_double_square}
    Let $k$ be a field of characteristic $\neq 2$ and $\alpha, \beta \in k$.
    \begin{itemize}
        \item there is $\gamma \in k$ with $\gamma = \alpha\beta$ and $2\gamma - \alpha - \beta$ square in $k$ if $-\alpha$ and $-\beta$ are squares in $k$.
        \item there is $\gamma \in k$ with $\gamma = \alpha(\alpha - \beta)$ and $2\gamma + 2\alpha - \beta$ square in $k$ if $\alpha$ and $\alpha - \beta$ are squares in $k$.
        \item there is $\gamma \in k$ with $\gamma = \beta(\beta - \alpha)$ and $2\gamma + 2\beta - \alpha$ square in $k$ if $\beta$ and $\beta - \alpha$ are squares in $k$.
    \end{itemize}
\end{lemma}
\begin{proof}
    Consider $\mu, \rho \in k$ with $\mu^2 = -\alpha$ and $\rho^2 = -\beta$.
    Then $\gamma := \mu\rho$ satisfies $\gamma^2 = \alpha\beta$ and
    \begin{equation*}
        (\mu + \rho)^2 = \mu^2 + 2\mu\rho + \rho^2 = 2\gamma -\alpha - \beta
    \end{equation*}
    Consider now $\mu, \rho \in k$ with $\mu^2 = \alpha$ and $\rho^2 = \alpha - \beta$.
    Then $\gamma := \mu\rho$ satisfies $\gamma^2 = \alpha(\alpha - \beta)$ and
    \begin{equation*}
        (\mu + \rho)^2 = \mu^2 + 2\mu\rho + \rho^2 = 2\gamma + 2\alpha - \beta
    \end{equation*}
    The third claim follows in the same way, by swapping $\alpha$ and $\beta$.
\end{proof}
\begin{example}
    Let $p$ be a prime.
    Taking the set of ``exhaustive cases'' given by $p(-1) = -p$ similar to the proof of Proposition~\ref{prop:subgroup_mod_every_p}.
    In other words, take $\alpha, \beta \in \Z$ such that $-\alpha, -p\beta$ and $\beta - \alpha$ are squares (in $\Z$).
    Then we find for any prime $q$
    \begin{itemize}
        \item If $p$ is a quadratic residue mod $q$, then $-\alpha$ and $-\beta$ are
        \item If $-1$ is a quadratic residue mod $q$, then $\alpha$ and $\alpha - \beta$ are
        \item If $-p$ is a quadratic residue mod $q$, then $\beta$ and $\beta - \alpha$ are
    \end{itemize}
    Hence, by Lemma~\ref{prop:condition_z4_subgroup_mod_p} and Lemma~\ref{prop:condition_double_square}, we see that every reduction $\tilde{E}(\F_q)$ of the Elliptic Curve $E: y^2 = x(x - \alpha)(x - \beta)$ (where $q$ is a prime of good reduction) contains a point of order 4.
    Thus $\Z/4\Z \times \Z/2\Z \hookrightarrow \tilde{E}(\F_q)$.
\end{example}
\begin{example}[2(ii) - Additional Examples]
    Consider the Elliptic Curve $E: y^2 = x(x + 4)(x + 3)$.
    Then $-(-4)$, $(-3)(-3)$ and $-3 - (-4) = 1$ are square, thus $\Z/4\Z \times \Z/2\Z \hookrightarrow \tilde{E}(\F_p)$ for every prime $p$ of good reduction.
    Furthermore, $E(\Q)$ does not have a point of order $4$ by Lemma~\ref{prop:condition_z4_subgroup_mod_p}, since $(-3)(-4) = 12$ and $(-3)(-3 - (-4)) = 3$ are no squares and also $2\gamma + 2(-4) - (-3) = -5 \pm 2 \cdot 2$ is not a square, where $\gamma^2 = (-4)(-4 - (-3)) = 4$.
\end{example}

\section{Question 3}
First, we first look at some basic transformations we can do to Weierstraß equations.
This will be our main toolkit for this exercise.
\begin{prop}[Weierstraß transformations]
    \label{prop:weierstrass_transformations}
    Let $E: y^2 + a_1 x y + a_3 y = x^3 + a_2 x^2 + a_4 x + a_6$ be an Elliptic Curve defined over $k$.
    There are three nice types of transformations
    \begin{description}
        \item[Translation] Let $P = (s, t) \in E(k)$ be a point. Then the isomorphism
        \begin{equation*}
            E \to E', \quad (x, y) \mapsto (x - s, y - t)
        \end{equation*}
        maps $P$ to $(0, 0)$ and the curve $E$ to an Elliptic Curve
        \begin{equation*}
            E': y^2 + a_1 x y + (a_3 + 2t + a_1 s) y = x^3 + (a_2 + 3 s) x^2 + (a_4 - 3 s^2 + 2 a_2 s - a_1 t) x
        \end{equation*}
        This is very useful to clear $a_6$ and continue working with the point $(0, 0)$.
        \item[Shearing] Let $r \in k$. Then the isomorphism
        \begin{equation*}
            E \to E', \quad (x, y) \mapsto (x, y - r x)
        \end{equation*}
        preserves $(0, 0)$ and maps $E$ to an Elliptic Curve
        \begin{equation*}
            E': y^2 + (a_1 + 2r) x y + a_3 y = x^3 + (a_2 + r^2 - a_1 r) x^2 + (a_4 - r a_3) x + a_6
        \end{equation*}
        This is very useful, as it does not change $a_3$ and $a_6$.
        \item[Scaling] Let $u \in k^*$. Then the isomorphism
        \begin{equation*}
            E \to E', \quad (x, y) \mapsto (u^2 x, y^3 y)
        \end{equation*}
        preserves $(0, 0)$ and maps $E$ to an Elliptic Curve
        \begin{equation*}
            E': y^2 + \frac {a_1} u x y + \frac {a_3} {u^3} = x^3 + \frac {a_2} {u^2} x^2 + \frac {a_4} {u^4} x + \frac {a_6} {u^6}
        \end{equation*}
        This is very useful, as it does not change which of the $a_1, ..., a_4, a_6$ are zero.
    \end{description}
\end{prop}
\begin{proof}
    Just plug the equation of the isomorphism into the equation defining the $E'$, and check that it is zero modulo the equation of $E$.
\end{proof}
\begin{corollary}
    \label{prop:point_normalization}
    Let $E$ be an Elliptic Curve defined over $k$ with a $k$-rational point $P$ that is not a 2-torsion point.
    Then there is an Elliptic Curve $E'$ and a linear isomorphism $\psi: E \to E'$ such that $P \mapsto (0, 0)$ and the tangent at $(0, 0)$ on $E'$ is given by the equation $y = 0$.
    
    Furthermore $E'$ is given by an equation of the form
    \begin{equation*}
        E': y^2 + a_1 x y + a_3 y = x^3 + a_2 x^2
    \end{equation*}
    and $[2](0, 0) = (-a_2, a_1 a_2 - a_3)$.
\end{corollary}
\begin{proof}
    After a translation by $-P$, we may assume that
    \begin{equation*}
        E: y^2 + a'_1 x y + a'_3 y = x^3 + a'_2 x^2 + a'_4 x
    \end{equation*}
    and $P = (0, 0)$.
    Now observe that if $a'_3 = 0$, the line $x = 0$ through $(0, 0)$ and $\O$ meets $E$ in $(0, 0)$ with multiplicity 2, and so $(0, 0) + \O = (0, 0)$, contradicting the assumption that $P$ is not a 2-torsion point.
    Thus $a'_3 \neq 0$ and a shearing with $r = a'_4 / a'_3$ maps $E$ to
    \begin{equation*}
        E': y^2 + a_1 x y + a_3 y = x^3 + a_2 x^2
    \end{equation*}
    Note that the tangent at $(0, 0)$ now has slope $0$, i.e. is given by $y = 0$.
    Furthermore, the third point of intersection of the tangent and $E$ is $(-a_2, 0)$.
    The line through $\O$ and $(-a_2)$ is given by $x = -a_2$ and its third point of intersection with $E$ is then $(-a_2, a_1a_2 - a_3)$. 
\end{proof}
I came up with the next lemma to make my first proof of 3(i) work.
Since then, I have found a simpler proof that does not require the lemma anymore, but I found it beautiful and did not want to delete it.
\begin{lemma}
    \label{prop:share_subgroup_equal}
    Let $E, E'$ be Elliptic Curves defined over any field $k$, and assume they share a cyclic subgroup of order $n \geq 5$\footnote{Technically, we can also allow infinite order here.}.
    With this, we mean there is a point $P \in E \cap E'$ of order $n$ such that
    \begin{equation*}
        G := \langle P \rangle_E \subseteq E' \quad \text{and} \quad \restr{+_E}{G \times G} = \restr{+_{E'}}{G \times G}
    \end{equation*}
    Then $E = E'$ (in the sense that they have the same defining equation).
\end{lemma}
\begin{proof}
    Consider some point $[i] P = (a, b) \in E \cap E'$.
    With $P' = (a', b') := -[2i] P \in E \cap E'$ we see that $P + P + P' = \O$ and so $P, P, P'$ are colinear\footnote{We mean that the line through $P$ and $P'$ meets $E$ resp. $E'$ at $P$ with multiplicity 2.}.
    In particular, the tangent on $E$ resp. on $E'$ at $P \in E \cap E'$ both have the slope $(b - b')/(a - a')$\footnote{Or infinity if $a = a'$, but importantly, the slopes are equal.}.

    Since $P$ has order at least 5, observe that $P, [2]P, [3]P, [4]P$ and $[5]P$ are all different.
    Furthermore, since $E$ and $E'$ have the same tangent slope at each $[i] P$, note that $E$ meets $E'$ in $[i] P$ with multiplicity 2.
    So $E$ meets $E'$ in at least 10 points (counting multiplicity), which is greater than the product of their degrees $9 = 3 \cdot 3$.
    By Bezout's theorem \cite[Corollary I.7.8]{hartshorne}, it follows that $E$ and $E'$ share an irreducible component of dimension $\geq 1$, but since both are Elliptic Curves, they are irreducible of dimension $1$ and so $E = E'$.
\end{proof}
\begin{prop}[3(i)]
    Let $E$ be an Elliptic Curve defined over $\Q$.
    Then $E(\Q)$ has a nontrivial 5-torsion point if and only if $E$ is isomorphic (over $\Q$) to an Elliptic Curve given by an equation of the form
    \begin{equation*}
        y^2 + (v + 1)x y + v y = x^3 + v x^2
    \end{equation*}
\end{prop}
\begin{proof}
    By Corollary~\ref{prop:point_normalization}, we can assume that $E$ is given as
    \begin{equation*}
        E: y^2 + a_1 x y + a_3 y = x^3 + a_2 x^2
    \end{equation*}
    and $(0, 0)$ is a 5-torsion point of $E$ such that the tangent at $(0, 0)$ on $E$ is given by $y = 0$.
    Further, have $[2](0, 0) = (-a_2, a_1 a_2 - a_3)$.
    After applying a scaling, assume further that $a_2 = a_3$.
    Now define $\beta = a_2(a_1 - 1)$.
    Thus $[2](0, 0) = (-a_2, \beta)$.
    By computing the third point of intersection between $E$ and the lines $x = 0$ resp. $x = -a_2$ we see that 
    \begin{equation*}
        [3](0, 0) = -[2](0, 0) = (-a_2, 0), \quad [4](0, 0) = (0, -a_3)
    \end{equation*}
    Now consider the tangent at $[2](0, 0) = (-a_2, \beta)$.
    It has slope
    \begin{align*}
        \lambda =& \frac {3 a_2^2 - 2 a_2^2 - a_1 \beta} {2\beta - a_1 a_2 + a_3} = \frac {a_2^2 - a_1\beta} \beta = \frac {a_2^2 - a_1a_2(a_1 - 1)} \beta = \frac {(a_2 - a_1(a_1 - 1))a_2} {a_2(a_1 - 1)} \\
        =& \frac {a_2 - a_1(a_1 - 1)} {a_1 - 1} = \frac {a_2} {a_1 - 1} - a_1
    \end{align*}
    Since $[4](-a_2, \beta) = [4](0, 0) = -(0, 0)$, observe that $(0, 0)$ must be a point on the tangent $y = \lambda (x + a_2) + \beta$.
    Thus $\lambda a_2 + \beta = 0$ and so
    \begin{equation*}
        \frac {a_2^2} {a_1 - 1} - a_1 a_2 + a_2(a_1 - 1) = 0
    \end{equation*}
    Clearly $a_2 \neq 0$ and thus
    \begin{equation*}
        0 = a_2 - a_1 (a_1 - 1) + (a_1 - 1)^2 = a_2 - a_1 - 1
    \end{equation*}
    So $a_1 = a_2 + 1$ and the claim follows with $v = a_2 = a_3$.
\end{proof}
In the lecture, it was mentioned that a theorem of Mazur states that the torsion group of an Elliptic Curve $E$ defined over $\Q$ has one of the following forms
\begin{itemize}
    \item $\Z/n\Z$ for $n \in \{ 1, ..., 10, 12 \}$
    \item $\Z/n\Z \oplus \Z/2\Z$ for $n \in \{ 2, 4, 6, 8 \}$
\end{itemize}
Hence, there are not many $N$ for which a similar idea can work.
First, note that the case $N = 2$ is easy.
\begin{prop}[3(ii) - $2$-torsion points]
    Let $E$ be an Elliptic Curve defined over $\Q$.
    Then $E(\Q)$ has a nontrivial 2-torsion point if and only if $E$ is isomorphic (over $\Q$) to an Elliptic Curve given by an equation of the form
    \begin{equation*}
        y^2 = x^3 + a_2 x^2 + a_4 x
    \end{equation*}
\end{prop}
\begin{proof}
    Assume $E: y^2 = x^3 + a_2 x^2 + a_4 x$ is an Elliptic Curve.
    Then clearly $(0, 0) \in E(\Q)$ and $(0, 0) = -(0, 0)$, so $(0, 0)$ is nontrival 2-torsion point.
    Hence, there is a nontrivial 2-torsion point in $E'(\Q)$ for all Elliptic Curves $E'$ that are isomorphic (over $\Q$) to $E$.

    Conversely, let $E$ be an Elliptic Curve with a nontrivial 2-torsion point in $E(\Q)$.
    Note that $E$ is isomorphic to an Elliptic Curve 
    \begin{equation*}
        E': y^2 = x^3 + a_2 x^2 + a_4 x + a_6
    \end{equation*}
    as this holds for every Elliptic Curve.
    Now let $(\alpha, \beta) \in E'(\Q)$ be a nontrival 2-torsion point.
    Thus $-(\alpha, \beta) = (\alpha, \beta)$, so $\beta = 0$ and $\alpha^3 + a_2 \alpha^2 + a_4 \alpha + a_6 = 0$.
    Now consider the isomorphism
    \begin{equation*}
        E' \to E'', \quad (x, y) \mapsto (x - \alpha, y)
    \end{equation*}
    where
    \begin{equation*}
        E'': y^2 = x^3 + (3\alpha + a_2) x^2 + (3\alpha^2 + 2\alpha a_2 + a_4)x + \underbrace{\alpha^3 + \alpha^2a_2 + \alpha a_4 + a_6}_{= 0}
    \end{equation*}
    Observe that $E''$ is of the described form, and the claim follows. 
\end{proof}
The case $N = 3$ is slightly more interesting.
Our approach is as follows:
\begin{itemize}
    \item Apply a translation to get an isomorphic curve whose (nontrivial) 3-torsion point is $(0, 0)$ and the tangent is given by $y = 0$.
    \item Observe that $(0, 0)$ being a 3-torsion point is equivalent to the fact that the tangent at $E$ through $(0, 0)$ meets $E$ at $(0, 0)$ with multiplicity three.
    \item Show that after a scaling, the resulting equation is nice.
\end{itemize}
Now we get
\begin{prop}[3(ii) - 3-torsion points]
    Let $E$ be an Elliptic Curve defined over $\Q$.
    Then $E(\Q)$ has a nontrivial 3-torsion point if and only if $E$ is isomorphic (over $\Q$) to an Elliptic Curve given by an equation of the form
    \begin{equation*}
        y^2 + x y + v y = x^3
    \end{equation*}
\end{prop}
\begin{proof}
    By Corollary~\ref{prop:point_normalization} we can assume wlog that
    \begin{equation*}
        E: y^2 + a_1 x y + a_3 y = x^3 + a_2 x^2
    \end{equation*}
    and $(0, 0) \in E(\Q)$ is a nontrivial 3-torsion point.

    Now note that the tangent at $(0, 0)$ has slope $m = 0/a_3 = 0$.
    Hence, since $-[2](0, 0) = (0, 0)$ is the third point of intersection of $E$ and the tangent, we see that the tangent meets $E$ at $(0, 0)$ with multiplicity three.
    Thus
    \begin{equation*}
        x^3 + a_2 x^2 
    \end{equation*}
    must already be $x^3$ and thus have $a_2 = 0$.
    Finally, apply a scaling with $u = a_1$ (note that $a_1 \neq 0$, otherwise the curve would be singular) and find that $E$ is isomorphic to the curve
    \begin{equation*}
        E': y^2 + x y + v y = x^3
    \end{equation*}
    where $v = a_3 / (a_1)^3$.

    Conversely, assume that $E$ is isomorphic to an Elliptic Curve of the above form, so wlog
    \begin{equation*}
        E: y^2 + x y + v y = x^3
    \end{equation*}
    We show that $(0, 0) \in E(\Q)$ has order 3.
    The tangent at $(0, 0)$ has slope $0$, so is given by the line $y = 0$.
    Plugging this in yields $x^3 = 0$, and so the third point of intersection with $E$ is $(0, 0)$.
    
    Now consider the line through $\O$ and $(0, 0)$, which is given by $x = 0$.
    Plugging this in yields $y^2 + v y = 0$ and so the third point of intersection is $(0, -v)$.
    Now note that $(0, 0)$, $(0, -v)$ and $\O$ are colinear, so $(0, 0) + (0, -v) + \O = \O$, hence $(0, 0) = -[2](0, 0)$ has order 3.
\end{proof}
A similar approach works also for 4-torsion points.
\begin{prop}[3(ii) - 4-torsion points]
    Let $E$ be an Elliptic Curve defined over $\Q$.
    Then $E(\Q)$ has a nontrivial 4-torsion point if and only if $E$ is isomorphic (over $\Q$) to an Elliptic Curve given by an equation of the form
    \begin{equation*}
        y^2 + x y + v y = x^3 + v x^2
    \end{equation*}
\end{prop}
\begin{proof}
    Again, by Corollary~\ref{prop:point_normalization}, assume wlog that
    \begin{equation*}
        E: y^2 + a_1 x y + a_3 y = x^3 + a_2 x^2
    \end{equation*}
    and $(0, 0) \in E(\Q)$ is a nontrivial 4-torsion point.
    Find that $[2](0, 0) = (-a_2, \beta)$ where $\beta = a_1 a_2 - a_3$.
    The tangent at $(-a_2, \beta)$ must have the equation $x = -a_2$ since $(-a_2, \beta)$ is 2-torsion by assumption.
    Thus
    \begin{equation*}
        0 = 2\beta - a_1a_2 + a_3 = \beta
    \end{equation*}
    and so $a_1a_2 = a_3$.
    By scaling with $a_1$ (which is nonzero, otherwise $a_3 = 0$ and the curve is singular), observe that $E$ is isomorphic to the curve
    \begin{equation*}
        E': y^2 + x y + v y = x^3 + v x^2
    \end{equation*}
    where $v = a_3 / a_1^3 = a_2 / a_1^2$.

    Conversely, assume that $E$ is isomorphic to an Elliptic Curve of the above form, so wlog
    \begin{equation*}
        E: y^2 + x y + v y = x^3 + v x^2
    \end{equation*}
    We show that $(0, 0) \in E(\Q)$ has order $4$.
    The tangent has slope $0$, so is given by the line $y = 0$.
    The third point of intersection with $E$ is now $(-v, 0)$.
    Note that the line through $(-v, 0)$ and $\O$ has the equation $x = -v$ and meets $E$ at $(-v, 0)$ with multiplicity 2.
    It follows that $(-v, 0)$ is a 2-torsion point, and so $E(\Q)$ has the point $(0, 0)$ of order 4.
\end{proof}
\begin{example}[3(ii) - Additional Examples]
    Consider the Elliptic Curve
    \begin{equation*}
        E: y^2 + x y + y = x^3
    \end{equation*}
    defined over $\Q$.
    Note that the reduction $\tilde{E}$ modulo $3$ is still an Elliptic Curve (If $3x^2 + y = 2y + 1 + x = 0$ then $y^2 - y - 1 = 0$, so $y = -1$ and $x = 1$. However $(1, -1) \notin \tilde{E}(\F_3)$).
    Trying all values in $\F_3^2$, we find
    \begin{equation*}
        \tilde{E}(\F_3) = \{ (0, 0), (0, 2), \O \} 
    \end{equation*}
    Clearly $E(\Q)$ has the 3-torsion point $(0, 0)$, and since $E_{\mathrm{tors}}(\Q) \hookrightarrow \tilde{E}(\F_3)$ we see that
    \begin{equation*}
        E_{\mathrm{tors}}(\Q) = \{ (0, 0), (0, -1), \O \}
    \end{equation*}
\end{example}

\section{Question 4}
Let $S = \{ x^2 \ | \ x \in \Q^* \}$.
\begin{example}[4(i)]
    \label{ex:4i}
    The Elliptic Curve $E: y^2 = x(x + 6x + 1)$ has rank $0$.
\end{example}
\begin{proof}
    As in the lecture, consider
    \begin{align*}
        &E': y^2 = x(x^2 - 12x + 32) \\
        &\phi: E \to E', \quad (u, v) \mapsto \left( \frac {y^2} {x^2}, \ y \frac {x^2 - 1} {x^2} \right) \\
        &\hat{\phi}: E' \to E, \quad (u, v) \mapsto \left( \frac {y^2} {4x^2}, \ y \frac {x^2 - 1} {8x^2} \right) \\
        &q: E'(\Q)/\phi(E(\Q)) \to \Q^*/S, \quad \overline{(u, v)} \mapsto \begin{cases}
            \overline{u} & \text{if $u \neq 0$} \\
            \overline{32} & \text{if $u = 0$}
        \end{cases} \\
        &\hat{q}: E(\Q)/\phi(E'(\Q)) \to \Q^*/S, \quad \overline{(u, v)} \mapsto \begin{cases}
            \overline{u} & \text{if $u \neq 0$} \\
            \overline{1} & \text{if $u = 0$}
        \end{cases}
    \end{align*}
    To find the rank, we proceed as in the lecture (technically, use \cite[Lemma 6.6]{lecture}).

    \paragraph{Find $E'(\Q)/\phi(E(\Q))$} Consider $r \divides 32$ square-free, i.e. $r \in \{ \pm 1, \pm 2 \}$.

    For $r = 2$, have that $(l, m, n) = (2, 1, 0)$ solves
    \begin{equation*}
        2l^4 - 12l^2m^2 + 16m^4 = n^2
    \end{equation*}
    and indeed we find $(8, 0) \in E'(\Q)$.

    For $r = -1$, note that
    \begin{equation*}
        -l^4 - 12l^2m^2 - 32m^4 = n^2
    \end{equation*}
    has no nontrivial solutions in $\Q$, as the left-hand side is always $\leq 0$ and the right-hand side is $\geq 0$.

    Since $-2 = -1 \cdot 2$, we see that $\im(q) = \langle 2 \rangle$ and $E'(\Q)/\phi(E(\Q)) = \langle (8, 0) \rangle$.

    \paragraph{Find $E(\Q)/\hat{\phi}(E'(\Q))$} Consider $r \divides 1$ square-free, i.e. $r \in \{ \pm 1 \}$.

    For $r = -1$, have that $(l, m, n) = (1, 1, 2)$ solves
    \begin{equation*}
        -l^4 + 6l^2m^2 - m^4 = n^2
    \end{equation*}
    and indeed find $(-1, 2) \in E(\Q)$.

    Thus find $\im(\hat{q}) = \langle -1 \rangle$ and $E(\Q)/\hat{\phi}(E'(\Q)) = \langle (-1, 2) \rangle$.

    \paragraph{Find the rank of $E$} By the above two steps, we see that
    \begin{equation*}
        E(\Q)/[2]E(\Q) = \langle (-1, 2), \hat{\phi}((8, 0)) \rangle = \langle (-1, 2), (0, 0) \rangle
    \end{equation*}
    Now observe that $[2](-1, 2) = (0, 0)$ and $[2](0, 0) = \O$.
    Hence $E(\Q) = E_{\mathrm{tors}}(\Q)$ and the rank is 0 as claimed.
\end{proof}
To find an example with rank two, it seems like a good way to take a curve with many rational points.
Further requiring those points to be non-integral increases our chance, as that way, they cannot be torsion points.
However, playing around with this method did not yield a nice curve where we can compute the rank.
So lets fall back to brute force, which works quite well here as this condition is very easy to test via a computer.
\begin{example}[4(ii)]
    \label{ex:special_curve}
    Consider the curve
    \begin{equation*}
        E: y^2 = x(x^2 + 47 x + 30)
    \end{equation*}
    Clearly $30 = 2 \cdot 3 \cdot 5$ has at least 3 different prime factors.
    We want to compute the rank of $E$.
    As always, have
    \begin{equation*}
        E': y^2 = x(x^2 - 92 x + 2089)
    \end{equation*}

    \paragraph{Find $E'(\Q)/\phi(E(\Q))$} Have $b_1 = 2089$ is prime, so consider $r \in \{ \pm 1, \pm 2098 \}$.
    First note that for $r < 0$, the equation
    \begin{equation*}
        r l^4 - 92 l^2 m^2 + \frac {2089} r m^4 = n^2
    \end{equation*}
    has no real nontrivial solutions, as the left-hand side is $< 0$ and the right-hand side is $\geq 0$.
    Hence, it is left to consider $r = 2089$.
    Notice that the equations
    \begin{equation*}
        2089 l^4 - 92 l^2 m^2 + m^4 = n^2
    \end{equation*}
    has the solution $(l, m, n) = (0, 1, 1)$ and so $E'(\Q)/\phi(E(\Q)) = \langle (0, 0) \rangle$.

    \paragraph{Find $E(\Q)/\hat{\phi}(E'(\Q))$} Have $b = 30 = 2 \cdot 3 \cdot 5$, so consider
    \begin{equation*}
        r \in \{ \pm 1, \pm 2, \pm 3, \pm 5, \pm 6, \pm 10, \pm 15, \pm 30 \}
    \end{equation*}

    The equation
    \begin{equation*}
        -l^4 + 47 l^2 m^2 - 30 m^4 = n^2
    \end{equation*}
    has the solution $(l, m, n) = (1, 1, 4)$ which gives a point $(-1, 4) \in E(\Q)$.

    The equation
    \begin{equation*}
        2l^4 + 47 l^2 m^2 + 15 m^4 = n^2
    \end{equation*}
    has the solution $(l, m, n) = (1, 1, 8)$ which gives a point $(2, 8) \in E(\Q)$.

    The equation
    \begin{equation*}
        3l^4 + 47 l^2 m^2 + 10 m^4 = n^2
    \end{equation*}
    has the solution $(l, m, n) = (3, 1, 26)$ which gives a point $(27, 234) \in E(\Q)$.

    The equation
    \begin{equation*}
        5l^4 + 47 l^2 m^2 + 6m^4 = n^2
    \end{equation*}
    has the solution $(l, m, n) = (1, 2, 17)$ which gives a point $(\frac 5 4, \frac {85} 8)$.

    Since $\im(\hat{q})$ is a group and $-1, 2, 3, 5$ clearly generate
    \begin{equation*}
        \{ \pm 1, \pm 2, \pm 3, \pm 5, \pm 6, \pm 10, \pm 15, \pm 30 \} \subseteq \Q^*/S
    \end{equation*}
    we already see that $\im(\hat{q}) = \langle -1, 2, 3, 5 \rangle$.

    \paragraph{Find the rank of $E$} Combining the above, we see that
    \begin{equation*}
        E(\Q)/[2]E(\Q) = \langle (-1, 4), (2, 8), (27, 234), \left(\frac 5 4, \frac {85} 8\right) \rangle
    \end{equation*}
    since $\hat{\phi}((0, 0)) = \O$.
    This further shows that $E(\Q)/[2]E(\Q) \cong (\Z/2\Z)^4$.
    Note that $x^2 + 47 x + 30$ has no rational root, so $(0, 0)$ is the only nontrivial 2-torsion points and thus
    \begin{equation*}
        E_{\mathrm{tors}}(\Q)/[2]E(\Q) \cong \Z/2\Z
    \end{equation*}
    Hence we see that the rank of $E$ is $4 - 1 = 3 \geq 2$.
\end{example}
Note that the computer also found other, similarly special curves, given by equations\footnote{I should have guessed that there is a solution involving 42.}
\begin{align*}
    E_1: y^2 = x(x^2 + 59 x^2 + 42) \\
    E_2: y^2 = x(x^2 + 83 x^2 + 78)
\end{align*}

\begin{prop}[4(iii)]
    \label{prop:rank_bound}
    Let $E: y^2 = x(x^2 + a x + b)$ be an Elliptic Curve such that $b(a^2 - 4b)$ has exactly $k$ prime factors.
    Then $\mathrm{rank}(E) \leq 2k$.
    Furthermore, we have
    \begin{itemize}
        \item If $a \leq 0, b \geq 0$, then $\mathrm{rank}(E) \leq 2k - 1$
        \item If $a \perp b$ are coprime, then $\mathrm{rank}(E) \leq k$
        \item If $a \perp b$ and $a \leq 0, b \geq 0$, then $\mathrm{rank}(E) \leq k - 1$
    \end{itemize}
    Note that if one of the additional conditions is fulfilled for $a_1, b_1$, then we get the corresponding bound for $\mathrm{rank}(E') = \mathrm{rank}(E)$ 
    (isogenous curves have the same rank, as isogenies are group homomorphisms with finite kernel, see \cite[Thm III.4.8]{silverman} and \cite[Corollary III.4.9]{silverman}).
\end{prop}
\begin{proof}
    Use $a_1, b_1, E', \phi, \hat{\phi}, q, \hat{q}$ as in the lecture.
    Let $l$ denote the number of distinct prime factors of $a^2 - 4b = b_1$ and $m$ denote the number of distinct prime factors of $b$.
    As shown in the lecture, have that $E'(\Q)/\phi(E(\Q)) \cong \im(q)$ and if $\overline{r} \in \im(q)$ with $r \in \Z$ square-free, then $r \divides b_1$.
    Thus
    \begin{equation*}
        \#(E'(\Q)/\phi(E(\Q))) = \#\im(q) \leq \#\{ r \divides b_1 \ | \ r \in \Z \ \text{square-free}\}
    \end{equation*}
    Now observe that there is a bijection
    \begin{equation*}
        \mathfrak{P}(\{ -1 \} \cup \{ p \divides b_1 \ | \ p \ \text{prime}\}) \to \{ r \divides b_1 \ | \ r \in \Z \ \text{square-free} \}, \quad M \mapsto \prod_{x \in M} x
    \end{equation*}
    and so
    \begin{equation*}
        \#(E'(\Q)/\phi(E(\Q))) \leq 2^{l + 1}
    \end{equation*}
    Note that the map $\hat{\phi}$ is a group homomorphism with kernel of size 2, and therefore we find
    \begin{equation*}
        \#(\hat{\phi}(E'(\Q))/[2]E(\Q)) \leq 2^l
    \end{equation*}
    Similarly, find
    \begin{equation*}
        \#(E(\Q)/\hat{\phi}(E'(\Q))) \leq 2^{m + 1}
    \end{equation*}
    Since there is a surjection
    \begin{equation*}
        E(\Q)/\hat{\phi}(E'(\Q)) \oplus \hat{\phi}(E'(\Q))/[2]E(\Q) \to E(\Q)/[2]E(\Q)
    \end{equation*}
    we see that
    \begin{equation}
        \#(E(\Q)/[2]E(\Q)) \leq 2^l \cdot 2^{m + 1} = 2^{l + m + 1} \label{eq:general_bound}
    \end{equation}
    Finally, note that
    \begin{equation*}
        E(\Q)/[2]E(\Q) \cong E_{\mathrm{tors}}(\Q)/[2]E(\Q) \oplus (\Z/2\Z)^{\mathrm{rank}(E)}
    \end{equation*}
    and thus $\Z/2\Z \hookrightarrow E_{\mathrm{tors}}(\Q)/[2]E(\Z)$ (there is the nontrivial 2-torsion point $(0, 0)$)
    This yields
    \begin{equation*}
        \mathrm{rank}(E) \leq \log_2(\#(E(\Z)/[2]E(\Q)) / 2) \leq \log_2(2^{l + m}) = l + m \leq 2k
    \end{equation*}

    \paragraph{Assume $a \leq 0, b \geq 0$} Then the equation
    \begin{equation*}
        r l^4 + a l^2 m^2 + \frac b r = n^2
    \end{equation*}
    has no real nontrivial solutions for $r \leq 0$.
    Since the solutions are in 1-to-1 correspondence with $\im(q)$, we see that
    \begin{equation*}
        \#(E(\Q)/\hat{\phi}(E'(\Q))) = \#\im(\hat{q}) \leq \#\{ r \divides b \ | \ r \in \Z \ \text{positive, square-free} \}
    \end{equation*}
    There is a bijection
    \begin{equation*}
        \mathfrak{P}(\{ p \divides b \ | \ p \ \text{prime} \}) \to \{ r \divides b \ | \ r \in \Z \ \text{positive, square-free} \}, \quad M \mapsto \prod_{x \in M} x
    \end{equation*}
    Thus
    \begin{equation*}
        \#(E(\Q)/\hat{\phi}(E'(\Q))) \leq 2^m
    \end{equation*}
    As before find then (since $\hat{\phi}$ has a kernel of size 2)
    \begin{equation}
        \#(E(\Q)/[2]E(\Q)) \leq 2^l \cdot 2^m \label{eq:sign_bound}
    \end{equation}
    and again as before
    \begin{equation*}
        \mathrm{rank}(E) \leq \log_2(2^l \cdot 2^m / 2) = l + m - 1 \leq 2k - 1
    \end{equation*}

    \paragraph{Assume $a \perp b$} Then have that $b_1 = (a^2 - 4b) \perp b$ and thus we find that $l + m = k$.
    Equation~\ref{eq:general_bound} is
    \begin{equation*}
        \#(E(\Q)/[2]E(\Q)) \leq 2^l \cdot 2^{m + 1} = 2^{l + m + 1}
    \end{equation*}
    and so it follows
    \begin{equation*}
        \mathrm{rank}(E) \leq \log_2(2^{l + m + 1} / 2) = l + m = k
    \end{equation*}

    \paragraph{Assume $a \perp b$ and $a \leq 0, b \geq 0$} Now have both Equation~\ref{eq:sign_bound}
    \begin{equation*}
        \#(E(\Q)/[2]E(\Q)) \leq 2^l \cdot 2^m = 2^{m + l}
    \end{equation*}
    and $m + l = k$.
    As before, it follows
    \begin{equation*}
        \mathrm{rank}(E) \leq \log_2(2^{m + l} / 2) = m + l - 1 = k - 1
    \end{equation*}
\end{proof}
Now we want to examine if the above inequalities are sharp.
\begin{example}
    Consider the curve $E: y^2 = x(x^2 - 6x + 1)$, which satisfies $a \leq 0, b \geq 0$ and $a \perp b$.
    Furthermore, $b_1 = 6^2 - 4 = 32 = 2^5$ has only one prime factor.
    Thus Proposition~\ref{prop:rank_bound} yields that $\mathrm{rank}(E) \leq 0$, so $\mathrm{rank}(E) = 0$.
\end{example}
Note that also Example~\ref{ex:special_curve} gives an example for the sharpness of part (iii), as the dual curve
\begin{equation*}
    E': y^2 = x(x - 92 x + 2089)
\end{equation*}
satisfies $a \leq 0, b \geq 0$ and $a \perp b$, and indeed its rank is $\mathrm{rank}(E') = 3 = 4 - 1$ (note that $2089 \cdot 30$ has exactly $4$ prime factors).

The next example shows that also part (i) of Proposition~\ref{prop:rank_bound} is sharp.
\begin{example}
    Consider the curve $E: y^2 = x(x^2 + 8)$, which satisfies $a \leq 0, b \geq 0$.
    Furthermore, $b b_1 = 8(-4 \cdot 8) = -256$ has only one prime factor.
    Thus Proposition~\ref{prop:rank_bound} yields that $\mathrm{rank}(E) \leq 2 - 1 = 1$.
    We claim that $\mathrm{rank}(E) = 1$.

    As always, have the curve
    \begin{equation*}
        E': y^2 = x(x^2 - 32)
    \end{equation*}
    \paragraph{Find $E'(\Q)/\phi(E(\Q))$} Have $b_1 = -32$, so consider $r \in \{ \pm 1, \pm 2 \}$.
    
    The equation
    \begin{equation*}
        -l^4 + 32m^4 = n^2
    \end{equation*}
    has the solution $(l, m, n) = (2, 1, 4)$ which gives a point $(-4, 8) \in E'(\Q)$.

    The equation
    \begin{equation*}
        2l^4 -16m^4 = n^2
    \end{equation*}
    has the solution $(l, m, n) = (2, 1, 4)$ which gives a point $(8, 16) \in E'(\Q)$.

    Hence also $-2 = -1 \cdot 2 \in \im(q)$ and we see that
    \begin{equation*}
        E'(\Q)/\phi(E(\Q)) = \langle (-4, 4), (8, 16) \rangle
    \end{equation*}

    \paragraph{Find $E(\Q)/\hat{\phi}(E'(\Q))$} Have $b = 8$, so consider $r \in \{ \pm 1, \pm 2 \}$.

    The equation
    \begin{equation*}
        -l^4 - 8m^4 = n^2
    \end{equation*}
    has no solution in $\mathbb{R}$, thus no solution in $\Q$.

    The equation
    \begin{equation*}
        2l^4 + 4m^4 = n^2
    \end{equation*}
    has the solution $(l, m, n) = (0, 1, 2)$ which gives a point $(0, 0) \in E(\Q)$.

    Hence also $-2 = -1 \cdot 2 \notin \im(\hat{q})$ and we see that
    \begin{equation*}
        E(\Q)/\hat{\phi}(E'(\Q)) = \langle \hat{\phi}(-4, 8), \hat{\phi}(8, 16), (0, 0) \rangle = \langle (1, 3), (0, 0) \rangle
    \end{equation*}

    \paragraph{Find the rank of $E$} By the above, have
    \begin{equation*}
        E(\Q)/[2]E(\Q) = \langle (1, 3), (0, 0) \rangle
    \end{equation*}
    Note that the square of the $y$-coordinate $3^2 = 9$ does not divide $\Delta(E) = 4 \cdot 8^3 = 2^{11}$ and so $(1, 3)$ is not torsion by the Nagell-Lutz theorem \cite[Thm 5.4]{lecture}.
    So it is of infinite order, and we have indeed that $\mathrm{rank}(E) = 1$.
\end{example}

\section{Appendix}
The curves from Example~\ref{ex:special_curve} were found by the following python script.
\lstinputlisting[language = python]{4ii_computations.py}

\printbibliography

\end{document}