\documentclass{scrartcl}

\usepackage{graphicx}
\usepackage[utf8]{inputenc}
\usepackage[T1]{fontenc}
\usepackage{lmodern}
\usepackage[english]{babel}
\usepackage{amsmath}
\usepackage{amsthm}
\usepackage{mathtools}
\usepackage{amssymb}
\usepackage{listings}
\usepackage{xparse}
\usepackage{geometry}
\usepackage{enumerate}
\usepackage{tikz}
\usepackage{hyperref}
\usepackage[style=english]{csquotes}
\usepackage[language=english, backend=biber, style=alphabetic, sorting=nyt]{biblatex}

\hypersetup{
    colorlinks,
    linkcolor={red!50!black},
    citecolor={blue!50!black},
    urlcolor={blue!80!black}
}

\usetikzlibrary{babel, positioning, shapes.geometric, arrows, arrows.meta}
\addbibresource{bibliography.bib}

\title{Miniproject - Introduction to Schemes}
\author{Simon Pohmann}

\newcommand{\N}{\mathbb{N}}
\newcommand{\Z}{\mathbb{Z}}
\newcommand{\F}{\mathbb{F}}
\newcommand{\C}{\mathbb{C}}
\newcommand{\Q}{\mathbb{Q}}
\newcommand{\I}{\mathbb{I}}
\newcommand{\V}{\mathbb{V}}
\newcommand{\p}{\mathfrak{p}}
\newcommand{\m}{\mathfrak{m}}
\renewcommand{\a}{\mathfrak{a}}
\renewcommand{\b}{\mathfrak{b}}
\renewcommand{\m}{\mathfrak{m}}
\newcommand{\Set}{\mathrm{\textbf{Set}}}
\newcommand{\Aff}{\mathrm{\textbf{Aff}}}
\newcommand{\Sch}{\mathrm{\textbf{Sch}}}
\newcommand{\Ring}{\mathrm{\textbf{Ring}}}
\newcommand{\Ab}{\mathrm{\textbf{Ab}}}
\newcommand{\Top}{\mathrm{Top}}
\newcommand{\Spec}{\mathrm{Spec}}
\newcommand{\Proj}{\mathrm{Proj}}
\newcommand{\Quot}{\mathrm{Quot}}
\newcommand{\im}{\mathrm{im}}
\renewcommand{\O}{\mathcal{O}}
\DeclareMathOperator*{\colim}{colim}

\newcommand\restr[2]{{
    \left.\kern-\nulldelimiterspace
    #1
    \vphantom{\big|}
    \right|_{#2}
}}

\newtheorem{prop}{Proposition}
\newtheorem{theorem}[prop]{Theorem}
\newtheorem{lemma}[prop]{Lemma}
\newtheorem{corollary}[prop]{Corollary}

\theoremstyle{definition}
\newtheorem{problem}[prop]{Problem}
\newtheorem{alg}[prop]{Algorithm}
\newtheorem{definition}[prop]{Definition}
\newtheorem{example}[prop]{Example}
\newtheorem{remark}[prop]{Remark}
\newtheorem{reminder}[prop]{Reminder}

\begin{document}
\maketitle

\section{Definition of $\Proj$}

\begin{definition}
    A \emph{graded ring} $S$ is a ring $S$ with a decomposition $S = \oplus_{d \in \Z} S_d$ into groups $S_i \subseteq S$ (w.r.t. addition in $S$) such that $S_i S_j \subseteq S_{i + j}$ for all $i, j \in \Z$.
    If all $S_d = \{ 0 \}$ for $d < 0$, call $S$ \emph{naturally graded ring}.
    Write further $S_+ := \sum_{d \neq 0} S_d$.
    For a homogeneous $f \in S_d$ say that $\deg(f) := d$ is its \emph{degree}.

    An element $f \in S$ is called \emph{homogeneous} (of degree $n$), if $f \in S_n$.
    An ideal $I \leq S$ is called \emph{homogeneous}, if it has a set of homogeneous generators.
\end{definition}
\begin{definition}
    For a naturally graded ring $S$, define the set
    \begin{equation*}
        \Proj(S) := \{ \p \in \Spec(S) \ | \ \text{$\p$ homogeneous}, \ S_+ \not\subseteq \p \}
    \end{equation*}
    of homogeneous prime ideals not containing $S_+$.

    This becomes a topological space by endowing it with the \emph{Zariski-topology} on $\Proj(S)$, given by the open sets
    \begin{equation*}
        D_{\a} := \{ \p \in \Proj(S) \ | \ \a \not\subseteq \p \}
    \end{equation*}
    for any homogeneous ideal $\a \leq S$.
\end{definition}
From now on let $S$ be a naturally graded ring.
\begin{prop}
    The above definition is well-defined, i.e. the sets $D_\a$ indeed form a topology on $\Proj(S)$.
\end{prop}
\begin{proof}
    Clearly $\Proj(S) = D_{\langle 1 \rangle}$ and $\emptyset = D_{\langle 0 \rangle}$ are open.
    Furthermore, for open sets $D_\a$ and $D_\b$, have that
    \begin{equation*}
        D_\a \cap D_\b = \{ \p \in \Proj(S) \ | \ \a, \b \not\subseteq \p \} = \{ \p \in \Proj(S) \ | \ \a\b \not\subseteq \p \} = D_{\a\b}
    \end{equation*}
    This holds, as $\a, \b \not\subseteq \p$ implies that there are $f \in \a$ and $g \in \b$ with $f, g \notin \p$.
    However, then $fg \notin \p$ as $\p$ is prime.
    Obviously $\a\b$ is homogeneous, and so $D_\a \cap D_\b$ is open.

    Finally, given a collection $\mathcal{A}$ of homogeneous ideals in $S$, have that
    \begin{align*}
        \bigcup_{\a \in \mathcal{A}} D_\a =& \{ \p \in \Proj(S) \ | \ \exists \a \in \mathcal{A}:\ \a \not\subseteq \p \} = \{ \p \in \Proj(S) \ | \ \exists \a \in \mathcal{A} \exists f \in \a:\ f \notin \p \} \\
        =& \Bigl\{ \p \in \Proj(S) \ \Bigm| \ \exists f \in \sum_{\a \in \mathcal{A}} \a:\ f \notin \p \Bigr\} = D_{\b} \quad \text{for $\b = \sum_{\a \in \mathcal{A}} \a$}
    \end{align*}
    Clearly $\b$ is again homogeneous, and so $\bigcup_{\a \in \mathcal{A}} D_\a$ is open.
\end{proof}
\begin{prop}
    The sets $D_f := D_{\langle f \rangle}$ for homogeneous $f \in S$ form a basis of the topology on $\Proj(S)$.
\end{prop}
\begin{proof}
    Clearly $\langle f \rangle$ is a homogenous ideal, so $D_f$ is open.
    For any homogeneous ideal $\a = \langle f_i \ | \ i \in I \rangle$ with $f_i \in S$ homogeneous have
    \begin{equation*}
        \a = \bigcup_{i \in I} D_{f_i}
    \end{equation*}
    as $\a \not\subseteq \p$ implies there is some $g = \sum_{i \in I} g_i f_i \notin \p$, with $g_i \in S$.
    Hence, at least one $g_j f_j \notin \p$ and so $f_j \notin \p$, thus $\p \in D_{f_j}$.
    It follows that the $D_f$ generate the topology on $\Proj(S)$, so it is left to show that they are a basis.

    Consider $\p \in D_f \cap D_g$, so $f, g \notin \p$.
    Since $\p$ is prime, it follows that $fg \notin \p$ and so $D_{fg} \subseteq D_f \cap D_g$ is an open neighborhood of $\p$.
\end{proof}
\begin{lemma}
    Let $S$ be a graded ring (not necessarily naturally graded) and $T \subseteq S$ a multiplicative set consisting of homogeneous elements. 
    Then $T^{-1}S$ becomes a graded ring via
    \begin{equation*}
        \left(T^{-1}S\right)_d = \Bigl\{ \frac g h \in T^{-1}S \ \Bigm| \ \text{$g$ homogeneous with $\deg(g) - \deg(h) = d$} \Bigr\}
    \end{equation*}
\end{lemma}
\begin{proof}
    Clearly $(T^{-1}S)_i (T^{-1}S)_j \subseteq (T^{-1}S)_{i + j}$.
    To see that $(T^{-1}S)_d$ is a subgroup of $S$, consider $g/f, l/h \in (T^{-1}S)_d$.
    Now have
    \begin{equation*}
        \frac g f + \frac l h = \frac {gh + lf} {hf}
    \end{equation*}
    and by assumption, find $\deg(gh) = \deg(g) - \deg(h) = d + \deg(f) - d + \deg(l) = \deg(lf)$.
    So $gh + lf$ is homogeneous and we have
    \begin{equation*}
        \deg(gh + lf) - \deg(fh) = \deg(f) + \deg(l) - \deg(f) - \deg(h) = \deg(l) - \deg(h) = d
    \end{equation*}
    Thus $g/f + l/h \in (T^{-1}S)_d$.

    Finally, we show that $(T^{-1}S)_n \cap (T^{-1}S)_m = \{ 0 \}$ for $i \neq j$.
    Assume there is $g/f = l/h \in (T^{-1}S)_n \cap (T^{-1}S)_m$ with $\deg(g) - \deg(f) = n$ and $\deg(l) - \deg(h) = m$.
    Then there exists $t \in T$ such that
    \begin{align*}
        0 = t(gh - lf) =& tgh - tlf \quad \text{with}\ tgh \in S_{\deg(t) + \deg(g) + \deg(h)} \\
        &\text{and}\ tlf \in S_{\deg(t) + \deg(l) + \deg(f)} = S_{\deg(t) + \deg(g) + \deg(h) + (m - n)}
    \end{align*}
    If $n \neq m$, then $S_{\deg(t) + \deg(g) + \deg(h) + (m - n)} \cap S_{\deg(t) + \deg(g) + \deg(h)} = \{ 0 \}$ and so $tgh = tlf = 0$.
    Thus $th(g - 0) = 0$ and so $g/f = 0/1 = 0$.
    Thus $(T^{-1}S)_m \cap (T^{-1}S)_n = \{ 0 \}$.
\end{proof}
\begin{lemma}
    \label{prop:homogeneous_part_prime}
    Let $\p \leq S$ be a prime ideal.
    Then
    \begin{equation*}
        \p' := \langle f \in \p \ | \ \text{$f$ homogeneous} \rangle \leq S
    \end{equation*}
    is a (homogeneous) prime ideal.
\end{lemma}
\begin{proof}
    Consider $f, g \in S$ with $f g \in \p'$ and assume $f, g \notin \p'$.
    Write $f = \sum_d f_d$ and $g = \sum_d g_d$ with $f_d, g_d \in S_d$.
    So
    \begin{equation*}
        \sum_{i, j} f_i g_j = \sum_n \sum_{i + j = n} f_i g_j \in \p'
    \end{equation*}
    Since $\p'$ is homogeneous, it follows that
    \begin{equation*}
        \sum_{i + j = n} f_i g_j \in \p'
    \end{equation*}
    for all $n \in \Z$.

    Let now $d$ resp. $e$ be maximal such that $f_d \notin \p'$ resp. $g_e \notin \p'$.
    We have
    \begin{equation*}
        f_d g_e + \sum_{\substack{i + j = d + e\\(i, j) \neq (d, e)}} \underbrace{f_i g_j}_{\in \p'} = \sum_{i + j = d + e} f_i g_j \in \p'
    \end{equation*}
    and so $f_d g_e \in \p' \subseteq \p$.
    Since $\p$ is prime, it follows that $f_d \in \p$ or $g_e \in \p$.
    However both $f_d$ and $g_e$ are homogeneous, so $f_d \in \p'$ or $g_e \in \p'$, a contradiction.
\end{proof}
\begin{lemma}
    If $D_g \subseteq D_f$ then there is a homogeneous $h \in S$ such that $g^n = fh$ for some $n \in \N$.
\end{lemma}
\begin{proof}
    Assume not, then $f$ is not a unit in $S_g$.
    Hence, there is a maximal ideal $\m \leq S_g$ such that $f \in \m$.
    Now let $\p$ be the preimage of $\m$ under the localization map $S \to S_g$.
    Since $\m$ is maximal, we see that $\p$ is prime.

    Now apply Lemma~\ref{prop:homogeneous_part_prime} and see that also
    \begin{equation*}
        \p' = \langle f \in \p \ | \ \text{$f$ homogeneous} \rangle \subseteq \p
    \end{equation*}
    is prime.
    Furthermore, $g \notin \p$ and so $g \notin \p'$.
    wlog we have that $g \in S_+$, so $S_+ \not\subseteq \p'$.
    Since now $\p'$ is a homogeneous prime ideal, it follows that $\p' \in \Proj(S)$.

    Finally, observe that $f \in \p'$ as $f \in \p$ and $f$ is homogeneous.
    Hence, we have that $\p' \notin D_f$ and $\p' \in D_g$, which contradicts the assumption that $D_g \subseteq D_f$.
\end{proof}
The next proof works exactly as the corresponding one for $\Spec$ in the lecture.
\begin{prop}
    Let $B = \{ D_f \ | \ \text{$f \in S$ homogeneous}\}$.
    The functor
    \begin{align*}
        \mathcal{F}: \restr{\Top(\Proj(S))}{B} \to \Ring, \quad D_f &\mapsto (S_f)_0 \\
        (D_{fg} \subseteq D_f) &\mapsto \Bigl( \restr{\cdot}{D_{fg}}: \frac s {f^n} \mapsto \frac {sg^n} {(fg)^n} \Bigr)
    \end{align*}
    is a B-sheaf on $B$ (here $\Top(X)$ is the category given by the open sets of $X$ and their inclusion, as defined in the lecture).
\end{prop}
\begin{proof}
    Clearly, $\mathcal{F}$ is a functor and thus a presheaf.
    Hence, we have to show the local-to-global property.

    Let $D_f = \bigcup_{i \in I} D_{g_if}$ be a cover and $s_i \in \mathcal{F}(D_{g_if})$ such that
    \begin{equation*}
        \forall x \in D_{g_if} \cap D_{g_jf} \ \exists V \in B: \ V \subseteq D_{g_if} \cap D_{g_jf}, \ x \in V, \ \frac {s_i} 1 = \frac {s_j} 1 \in \mathcal{F}(V)
    \end{equation*}
    To show uniqueness, it suffices to show that if
    \begin{equation*}
        \restr{\alpha}{D_{h_i}} = \restr{\beta}{D_{h_i}} \quad \text{for any $h_i$ with}\ \Proj(S) = \bigcup_i D_{h_i}
    \end{equation*}
    then $\alpha = \beta$.

    By assumption, have $h_i^{n_i}(\alpha - \beta) = 0$ for each $i$, and wlog there are only finitely many $i$.
    Thus find $N = \max_i n_i \in \N$ and get that $h_i^N(\alpha - \beta) = 0$.
    Since $\bigcup_i D_{h_i} = \Proj(S)$ it follows that $\bigcup_i D_{h_i^N} = \Proj(S)$ and so $1 \in \langle h_i^N \ | \ i \rangle$.
    It follows that
    \begin{equation*}
        \alpha - \beta = 1(\alpha - \beta) \in \langle h_i^N \ | \ i \rangle (\alpha - \beta) = \langle h_i^N (\alpha - \beta) \ | \ i \rangle = \{ 0 \}
    \end{equation*}
    and so $\alpha = \beta$.

    Now we show existence.
    By the uniqueness above, it follows that
    \begin{equation*}
        \restr{s_i}{D_{fg_ig_j}} = \restr{s_i}{D_{fg_i} \cap D_{fg_j}} = \restr{s_j}{D_{fg_i} \cap D_{fg_j}} = \restr{s_j}{D_{fg_ig_j}}
    \end{equation*}
    wlog have again a finite cover, i.e. only finitely many $g_i$.
    Hence find an $N \in \N$ such that each $s_i = s'_i / (fg_i)^N$ with $s'_i \in S$ homogeneous.
    By possibly replacing $N$ with a bigger $N$, we can now assume that
    \begin{equation*}
        (f^2g_ig_j)^N \left( s'_i (fg_j)^N - s'_j (fg_i)^N \right) = 0 \quad \text{as $\restr{s_i}{D_{fg_ig_j}} = \restr{s_j}{D_{fg_ig_j}}$}
    \end{equation*}
    Now note that
    \begin{equation*}
        s_i = \frac {a_i} {b_i} \quad \text{with}\ a_i = s'_i(fg_i)^N, \ b_i = (fg_i)^{2N}
    \end{equation*}
    and
    \begin{equation*}
        a_i b_j - a_j b_i = s'_i (fg_i)^N (fg_j)^{2N} - s'_j (fg_j)^N (fg_i)^{2N} = \underbrace{(f^2g_ig_j)^N \left(s'_i (fg_j)^N - s'_j (fg_i)^N \right)}_{= 0}
    \end{equation*}
    Now observe that $D_{b_i} = D_{fg_i}$ and so $f^m \in \langle b_i \ | \ i \rangle$ for some $m$.
    Let $1 = \sum_i r_i b_i$ and get
    \begin{equation*}
        a_i f^m = \sum_l r_l b_l a_i = \sum_l r_l a_l b_i = b_i \sum_l r_l a_l
    \end{equation*}
    Note that $a_i, b_i, f$ are homogeneous, and so we can also choose $r_i$ to be homogeneous.
    Then find that $0 = \deg(s_i) = \deg(a_i) - \deg(b_i) = \deg(\sum r_l a_l) - m\deg(f)$.

    Thus
    \begin{equation*}
        s_i = s := \frac {\sum_l r_l a_l} {f^m} \in S_f
    \end{equation*}
    and since $\deg(\sum r_l a_l) = m\deg(f)$ we find that $s \in (S_f)_0 = \mathcal{F}(D_f)$.
    Clearly $\restr{s}{D_{fg_i}} = s_i$ and the claim follows.
\end{proof}
\begin{corollary}
    Hence we can (uniquely) extend the B-sheaf $\mathcal{F}$ to a sheaf $\O_{\Proj(S)}$ on $\Proj(S)$.
\end{corollary}
The next to lemmas are based on \cite[II.2.5]{hartshorne} and show that $(\Proj(S), \O_{\Proj(S)})$ is a scheme.
\begin{lemma}
    For $\p \in \Proj(S)$, the stalk
    \begin{equation*}
        \O_{\Proj(S), \p} = (T^{-1}S)_0
    \end{equation*}
    is a local ring, where $T = \{ f \notin \p \ | \ \text{$f$ homogeneous}\}$ contains all homogeneous elements not in $\p$.
\end{lemma}
\begin{proof}
    Consider the ideal
    \begin{equation*}
        \m = \Bigl\{ \frac f g \in (T^{-1}S)_0 \ \Bigm| \ f \in \p \Bigr\}
    \end{equation*}
    We claim this is the unique maximal ideal of $(T^{-1}S)_0$.

    First, note that $1 \notin (T^{-1}S)_0$ as otherwise, there would be $f/g \in T^{-1}S, f \in \p$ with $t(f - g) = 0$ for some $t \in T$.
    However then $tg = tf \in \p$, and so $g \in \p$ (as $t \notin \p$), contradicting $g \in T$.
    
    Now assume there is any ideal $\a$ such that $\a \setminus \m \neq \emptyset$, i.e. there is $f/g \in \a \setminus \m$.
    Then $f \in T$ as $f$ homogeneous and $f \notin \p$.
    Thus $g/f \in (T^{-1}S)_0$ and so $f/g \in (T^{-1}S)_0^*$, which implies $\a = \langle 1 \rangle$.
\end{proof}
\begin{lemma}
    For $f \in S_+$ homogeneous have that $(D_f, \restr{\O_{\Proj(S)}}{D_f})$ is an affine scheme.
\end{lemma}
\begin{proof}
    Let $R = (S_f)_0$.
    Consider the map
    \begin{equation*}
        f: D_f \to \Spec(R), \quad \p \mapsto \p S_f \cap (S_f)_0
    \end{equation*}
    Note that it is continuous, as the preimage of some basic open set $D_{g/f^n}$ is $D_{fg} \subseteq D_f$ open.
    Furthermore, $f$ has the inverse
    \begin{equation*}
        f^{-1}: \Spec(R) \to D_f, \quad \p \mapsto \p S_f
    \end{equation*}
    which is also continuous, as the preimage of some basic open set $D_{fg}$ is $D_{g^n/f^m}$ where $n\deg(g) = m\deg(f)$.
    Thus $f$ is a homeomorphism.

    Now consider the natural transformation
    \begin{equation*}
        \eta: \O_{\Spec R} \Rightarrow f_*\Bigl( \restr{\O_{\Proj(S)}}{D_f} \Bigr)
    \end{equation*}
    given on basic open sets by
    \begin{equation*}
        \eta_{D_{g/f^n}}: R_{g/f^n} \to (S_{fg})_0, \quad \frac {h/f^m} {g/f^n} \mapsto \frac {h f^n} {g f^m}
    \end{equation*}
    Clearly this is a ring isomorphism, so $\eta$ is a natural isomorphism.
    The claim follows.
\end{proof}
\begin{corollary}
    $(\Proj(S), \O_{\Proj(S)})$ is a scheme.
\end{corollary}

\printbibliography
\end{document}