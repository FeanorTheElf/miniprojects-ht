\documentclass{scrartcl}

\usepackage{graphicx}
\usepackage[utf8]{inputenc}
\usepackage[T1]{fontenc}
\usepackage{lmodern}
\usepackage[english]{babel}
\usepackage{amsmath}
\usepackage{amsthm}
\usepackage{mathtools}
\usepackage{amssymb}
\usepackage{listings}
\usepackage{xparse}
\usepackage{geometry}
\usepackage{enumerate}
\usepackage{tikz}
\usepackage{hyperref}
\usepackage[style=english]{csquotes}
\usepackage[language=english, backend=biber, style=alphabetic, sorting=nyt]{biblatex}

\hypersetup{
    colorlinks,
    linkcolor={red!50!black},
    citecolor={blue!50!black},
    urlcolor={blue!80!black}
}

\usetikzlibrary{babel, positioning, shapes.geometric, arrows, arrows.meta}
\addbibresource{bibliography.bib}

\title{Miniproject - Introduction to Schemes}
\author{Simon Pohmann}

\newcommand{\N}{\mathbb{N}}
\newcommand{\Z}{\mathbb{Z}}
\newcommand{\F}{\mathbb{F}}
\newcommand{\C}{\mathbb{C}}
\newcommand{\Q}{\mathbb{Q}}
\newcommand{\I}{\mathbb{I}}
\newcommand{\V}{\mathbb{V}}
\newcommand{\p}{\mathfrak{p}}
\newcommand{\m}{\mathfrak{m}}
\renewcommand{\a}{\mathfrak{a}}
\renewcommand{\b}{\mathfrak{b}}
\renewcommand{\m}{\mathfrak{m}}
\newcommand{\Set}{\mathrm{\textbf{Set}}}
\newcommand{\Aff}{\mathrm{\textbf{Aff}}}
\newcommand{\Sch}{\mathrm{\textbf{Sch}}}
\newcommand{\Ring}{\mathrm{\textbf{Ring}}}
\newcommand{\Ab}{\mathrm{\textbf{Ab}}}
\newcommand{\Top}{\mathrm{Top}}
\newcommand{\Spec}{\mathrm{Spec}}
\newcommand{\Proj}{\mathrm{Proj}}
\newcommand{\Quot}{\mathrm{Quot}}
\newcommand{\im}{\mathrm{im}}
\renewcommand{\O}{\mathcal{O}}
\DeclareMathOperator*{\colim}{colim}

\newcommand\restr[2]{{
    \left.\kern-\nulldelimiterspace
    #1
    \vphantom{\big|}
    \right|_{#2}
}}

\newtheorem{prop}{Proposition}
\newtheorem{theorem}[prop]{Theorem}
\newtheorem{lemma}[prop]{Lemma}
\newtheorem{corollary}[prop]{Corollary}

\theoremstyle{definition}
\newtheorem{problem}[prop]{Problem}
\newtheorem{alg}[prop]{Algorithm}
\newtheorem{definition}[prop]{Definition}
\newtheorem{example}[prop]{Example}
\newtheorem{remark}[prop]{Remark}
\newtheorem{reminder}[prop]{Reminder}

\begin{document}
\maketitle

\section{Definition of $\Proj$}

This section is based on \cite[II.2]{hartshorne}
\begin{reminder}
    A \emph{graded ring} $S$ is a ring $S$ with a decomposition $S = \oplus_{d \in \N} S_d$ into groups $S_i \subseteq S$ (w.r.t. addition in $S$) such that $S_i S_j \subseteq S_{i + j}$ for all $i, j \in \N$.
    Write further $S_+ := \sum_{d > 0} S_d$.

    An element $f \in S$ is called \emph{homogeneous} (of degree $n$), if $f \in S_n$.
    An ideal $I \leq S$ is called \emph{homogeneous}, if it has a set of homogeneous generators.
\end{reminder}
\begin{definition}
    For a graded ring $S$, define the set
    \begin{equation*}
        \Proj(S) := \{ \p \in \Spec(S) \ | \ \text{$\p$ homogeneous}, \ S_+ \not\subseteq \p \}
    \end{equation*}
    of homogeneous prime ideals not containing $S_+$.

    This becomes a topological space by endowing it with the \emph{Zariski-topology} on $\Proj(S)$, given by the open sets
    \begin{equation*}
        D_{\a} := \{ \p \in \Proj(S) \ | \ \a \not\subseteq \p \}
    \end{equation*}
    for any homogeneous ideal $\a \leq S$.
\end{definition}
From now on let $S$ be a graded ring.
\begin{prop}
    The above definition is well-defined, i.e. the sets $D_\a$ indeed form a topology on $\Proj(S)$.
\end{prop}
\begin{proof}
    Clearly $\Proj(S) = D_{\langle 1 \rangle}$ and $\emptyset = D_{\langle 0 \rangle}$ are open.
    Furthermore, for open sets $D_\a$ and $D_\b$, have that
    \begin{equation*}
        D_\a \cap D_\b = \{ \p \in \Proj(S) \ | \ \a, \b \not\subseteq \p \} = \{ \p \in \Proj(S) \ | \ \a\b \not\subseteq \p \} = D_{\a\b}
    \end{equation*}
    This holds, as $\a, \b \not\subseteq \p$ implies that there are $f \in \a$ and $g \in \b$ with $f, g \notin \p$.
    However, then $fg \notin \p$ as $\p$ is prime.
    Obviously $\a\b$ is homogeneous, and so $D_\a \cap D_\b$ is open.

    Finally, given a collection $\mathcal{A}$ of homogeneous ideals in $S$, have that
    \begin{align*}
        \bigcup_{\a \in \mathcal{A}} D_\a =& \{ \p \in \Proj(S) \ | \ \exists \a \in \mathcal{A}:\ \a \not\subseteq \p \} = \{ \p \in \Proj(S) \ | \ \exists \a \in \mathcal{A} \exists f \in \a:\ f \notin \p \} \\
        =& \Bigl\{ \p \in \Proj(S) \ \Bigm| \ \exists f \in \sum_{\a \in \mathcal{A}} \a:\ f \notin \p \Bigr\} = D_{\b} \quad \text{for $\b = \sum_{\a \in \mathcal{A}} \a$}
    \end{align*}
    Clearly $\b$ is again homogeneous, and so $\bigcup_{\a \in \mathcal{A}} D_\a$ is open.
\end{proof}
\begin{prop}
    The sets $D_f := D_{\langle f \rangle}$ for homogeneous $f \in S$ form a basis of the topology on $\Proj(S)$.
\end{prop}
\begin{proof}
    Clearly $\langle f \rangle$ is a homogenous ideal, so $D_f$ is open.
    For any homogeneous ideal $\a = \langle f_i \ | \ i \in I \rangle$ with $f_i \in S$ homogeneous have
    \begin{equation*}
        \a = \bigcup_{i \in I} D_{f_i}
    \end{equation*}
    as $\a \not\subseteq \p$ implies there is some $g = \sum_{i \in I} g_i f_i \notin \p$, with $g_i \in S$.
    Hence, at least one $g_j f_j \notin \p$ and so $f_j \notin \p$, thus $\p \in D_{f_j}$.
    It follows that the $D_f$ generate the topology on $\Proj(S)$, so it is left to show that they are a basis.

    Consider $\p \in D_f \cap D_g$, so $f, g \notin \p$.
    Since $\p$ is prime, it follows that $fg \notin \p$ and so $D_{fg} \subseteq D_f \cap D_g$ is an open neighborhood of $\p$.
\end{proof}
\begin{lemma}
    If $D_g \subseteq D_f$ then there is a homogeneous $h \in S$ such that $g^n = fh$ for some $n \in \N$.
\end{lemma}
\begin{proof}
    Assume not, then $f$ is not a unit in $S_g$.
    Thus $\langle f \rangle \subseteq \m$ for a maximal ideal $\m \leq S_g$.
    Now let $\p$ be the preimage of $\m$ under the localization map $S \to S_g$, so $g \in \p$ and $f \notin \p$.
    
\end{proof}
\begin{prop}
    Let $B = \{ D_f \ | \ \text{$f \in S$ homogeneous}\}$.
    The functor
    \begin{align*}
        \mathcal{F}: \restr{\Top(\Proj(S))}{B} \to \Ring, \quad D_f &\mapsto S_f \\
        (D_{fg} \subseteq D_f) &\mapsto \Bigl( \frac s {f^n} \mapsto \frac {sg^n} {(fg)^n} \Bigr)
    \end{align*}
    is a B-sheaf on $B$ (here $\Top(X)$ is the category given by the open sets of $X$ and their inclusion, as defined in the lecture).
\end{prop}

\printbibliography
\end{document}